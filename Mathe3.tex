\documentclass[a4paper,12pt,twoside]{article}
\usepackage{fourier}
\usepackage{polynom}
\usepackage[ngerman]{babel}
\usepackage[leqno,tbtags,nointlimits]{amsmath}
\usepackage{amssymb,amsthm,amsfonts}
\usepackage{graphicx}
\usepackage{ifthen}
%\usepackage{gauss}
\usepackage{tikz}
\usepackage{mathtools}
\usepackage{makeidx}
\usepackage{fancyhdr,lastpage}
\usepackage{enumerate}
\usepackage[onehalfspacing]{setspace}
\usepackage{mdsymbol}
\usepackage{marvosym}
\usepackage{cancel}
\usepackage{pgfplots}
\usepackage{color}
\usepackage{bigints}
\usepackage{array}
\usepackage{mdframed}
\usepackage{marginnote}
\usetikzlibrary{trees,automata,arrows,shapes,decorations.pathmorphing,matrix}
\pagestyle{fancy}
\usepackage{hyperref}
\fancyhf{} %--Clear all fields
\renewcommand\sectionmark[1]{ \markboth{\thesection\ \textsc{#1}}{}}
\fancyhead[LO,RE]{\small \leftmark}
\fancyhead[LE,RO]{ \rightmark}
\fancyfoot{} % clear all footer fields
\fancyfoot[LE,RO]{\thepage}
\newcommand{\cd}{\cdot}
\newcommand{\C}{\mathbb{C}}
\newcommand{\Z}{\mathbb{Z}}
\renewcommand{\i}{\item}
\newcommand{\U}{\mathcal{U}}
\newcommand{\N}{\mathbb{N}}
\newcommand{\R}{\mathbb{R}}
\newcommand{\raum}[1]{\left\langle#1\right\rangle}
\DeclarePairedDelimiter{\ceil}{\lceil}{\rceil}
\DeclarePairedDelimiter{\floor}{\lfloor}{\rfloor}
\usepackage[normalem]{ulem}
\usepackage{blkarray}
\usepackage{stmaryrd}
\usepackage{titletoc}
\newcommand{\abs}[1]{\lvert #1 \rvert}
\renewcommand\headrule{{\color{gray}%
\hrule height 2pt width\headwidth
\vspace{1pt}%
\hrule height 1pt width\headwidth
\vspace{-4pt}}}
\makeatletter
\newcommand{\resetHeadWidth}{\fancy@setoffs}
\makeatother
\newcommand{\limit}[1]{\displaystyle \lim_{#1}}
\usepgfplotslibrary{fillbetween}
\pgfplotsset{compat=1.9}
\newcommand\vektor[3]{\begin{pmatrix}
#1\\#2\\#3
\end{pmatrix}}
\newcommand\vektort[2]{\begin{pmatrix}
#1\\#2
\end{pmatrix}}
\newcommand\vektorf[4]{\begin{pmatrix}
#1\\#2\\#3\\#4
\end{pmatrix}}
\newcount\vectorcount
\renewcommand*\vector[1]{%
  \global\vectorcount#1
  \begin{pmatrix}
    \vectornext
}
\def\vectornext#1{%
  #1
  \global\advance\vectorcount-1
    \ifnum\vectorcount>0
      \\
      \expandafter\vectornext
    \else
      \end{pmatrix}
    \fi
}
\DeclareMathOperator{\Grad}{Grad}
\makeindex
\hypersetup{%
pdfborder = {0 0 0}
}
\begin{document}
\resetHeadWidth
\newpage
\tableofcontents
\listoffigures
\clearpage
\setcounter{section}{-1}
\begin{center}
\Huge Ende des SS 2015
\end{center}
\section{Der Vektorraum $\R^n$}
$n \in \N\quad \R^n = \left\{ \begin{pmatrix}
a_1\\
\vdots\\
a_n
\end{pmatrix} : a_1 \in \R \right\}$\\
\emph{Spaltenvektoren} der Länge $n:\: \begin{pmatrix}
a_1\\\vdots\\a_n
\end{pmatrix} = (a_1,\ldots,a_n)^t\\
a_1,\ldots,a_n$ \emph{Komponente} der Spaltenvektoren.\\
Wie bei Matrizen:\\
\begin{minipage}{.5\textwidth}\[ \begin{pmatrix}
a_1\\\vdots\\a_n
\end{pmatrix}+ \begin{pmatrix}
b_1\\\vdots\\b_n
\end{pmatrix} = \begin{pmatrix}
a_1 + b_1\\\vdots\\a_n + b_n
\end{pmatrix}\]
\end{minipage}%
\begin{minipage}{.5\textwidth}
(Multiplikation entspricht der Matrixmultiplikation und ist nicht möglich falls $n > 1$)
\end{minipage}\\
Multiplikation eines Spaltenvektors mit einer Zahl (\emph{Skalar})
\[ a \cd \begin{pmatrix}
a_1\\\vdots\\a_n
\end{pmatrix} = \begin{pmatrix}
aa_1\\\vdots\\aa_n
\end{pmatrix} \]
Addition+Abbildung : $\R^n \times \R^n \to \R^n$\\
$\R^n$ mit Addition und Multiplikation mit Skalaren : \emph{$\R$-Vektorraum}\\
Die Vektoren im $\R^1 (= \R),\R^2$ und $\R^3$ entsprechen Punkten auf der Zahlengerade, Ebene, dreidimensionalen Raums.
\begin{figure}[h!]
\centering
\caption{Ein Vektor dargestellt durch seinen Ortsvektor}
\begin{tikzpicture}
\begin{axis}[axis equal, axis y line = center, axis x line = center, xmax = 3 , ymax = 3, ymin = 0 , xmin = -0.5,xtick={0.1,2},ytick={2},xticklabels={$\begin{pmatrix}
0\\0
\end{pmatrix}$,$a_1$},yticklabels={$a_2$}]
\draw[->] (axis cs:0,0)--(axis cs: 2,2);
\end{axis}
\end{tikzpicture}
\end{figure}
Punkte des $\R^2,\R^3$ lassen sich identifizieren mit, {\em Ortsvektoren} Pfeile mit Beginn in 0 (Komp = 0) und Ende im entsprechenden Punkt\\
Addition von Spaltenvektoren entspricht der Addition von Ortsvektoren entsprechend der Parallelogrammregel.
\begin{figure}[h!]
\centering
\caption{Vektoraddition durch Parallelogrammbildung}
\begin{tikzpicture}
\begin{axis}[
axis x line=center,
axis y line=center,
axis equal,
ymin = -1,
xmin = -3,
ymax = 7,
xmax = 5.5,
xtick ={4},
xticklabels={a},
ytick ={1},
yticklabels={b}]
\addplot [black, mark = *, nodes near coords=,every node near coord/.style={anchor=180}] coordinates {( 4, 3)};
   \draw[->](axis cs:0,0)--(axis cs:3.88,2.88);
       \addplot [black, mark = *, nodes near coords=,every node near coord/.style={anchor=0}] coordinates {( -1, 1)};
              \draw[->](axis cs:0,0)--(axis cs:-0.88,0.88);
\addplot [black, mark = *, nodes near coords=,every node near coord/.style={anchor=0}] coordinates {( 3, 4)};
\addplot[mark=none, black] coordinates {(-1,1) (3,4)};
\addplot[mark=none, black] coordinates {(3,4) (4,3)};
\draw[-> ,red](axis cs:0,0)--(axis cs:2.95,3.88);
\end{axis}
\end{tikzpicture}\end{figure}
Multiplikation mit Skalaren a :\\ Streckung (falls $\abs{a} > 1$)\\ Stauchung (falls $0 \ge \abs{a} \ge 1$)\\
Richtungspunkt, falls $a < 0$
TODO: Steckung und Stauchung\\
\subsection{Satz (Rechenregeln in $\R^n$)}\label{sec:0.1}
Seien $u,v,w \in \R^n,a,b\in\R$ Dann gilt:\\
\begin{enumerate}[a)]
\item \begin{align}
u + (v + w) &= (u + v) + w \tag{1.1}\\
v + 0 = 0 + v &= v, \text{wobei } 0\ \textit{Nullvektor} \tag{1.2}\\
v + -v &= 0 \tag{1.3}\marginnote{$\R^n$ kommutative Gruppe}\\
u + v &= v + u \tag{1.4}\\
(a + b)v &= av + bv \tag{2.1}\\
a(u + v) &= au + av \tag{2.2}\\
(a \cd b)v &=a(bv)\tag{2.3}\\
1v &= v \tag{2.4}
\end{align}
\item $0 \cdot v = 0$ und $ a \cdot 0 = 0$\\
Beweis folgt aus entsprechenden Rechenregeln in 0
\end{enumerate}
\subsection{Definition}\label{sec:0.2}
Eine nicht-leere Teilmenge $\mathcal{U} \supset \R^n$ hei\ss t \emph{Unterraum} (oder \emph{Teilraum} von $\R^n$), falls gilt:
\begin{enumerate}[(1)]
\item $\forall u_1,\,u_2 \in \mathcal{U}:\: u_1 + u_2 \in \mathcal{U}$ (Abgeschlossenheit bezüglich +)
\item $\forall u \in \mathcal{U} \forall a \in \R:\: au \in \mathcal{U}
$(Abgeschlossenheit bezüglich Mult. mit Skalaren)
\end{enumerate}
$\mathcal{U}$ enthält Nullvektor \{0\} Unterraum von $\R^n$ (Nullraum)\\
$\R^n $ ist Unterraum von $\R$
\subsection{Beispiele}\label{sec:0.3}
\begin{enumerate}[a)]
\item $0 \ne v \in \R^2\quad G = \{ av: a \in \R \}$ ist Unterraum von $\R^n$\begin{minipage}{.3\textwidth}
($a_1v,a_2v \in G, (a_1 + a_2)v \in G$\quad2.1 in \ref{sec:0.2}\\
$av \in G, b \in \R (ba)v \in G$)
\end{minipage}\\
G = Ursprungsgerade durch $\begin{pmatrix}
0\\0
\end{pmatrix}$ und v = $\begin{pmatrix}
a_1\\a_2
\end{pmatrix}
n = 2$:
\begin{figure}[h!]
\centering
\caption{Gerade dargestellt durch Vektoren}
\begin{tikzpicture}
\begin{axis}[axis equal, axis y line = center, axis x line = center, xmax = 3 , ymax = 2, ymin = -2, xmin = -0.5,xtick={0.1,2},ytick={2},xticklabels={$\begin{pmatrix}
0\\0
\end{pmatrix}$,$a_1$},yticklabels={$a_2$}]
\draw[blue,thick] (axis cs:-2,-2)--(axis cs: 2,2);
\end{axis}
\end{tikzpicture}
\end{figure}
\item $v,w \in \R^n\\
E = \{ av + bw : a,b \in \R \}$ ist Unterraum von $\R^n\\
v = o, w = o :\: E = \{ o \}\\
v \ne o\quad w \not\in \{ av:\: a \in \R \}\\
E = \R^2 $
$n = 3:\:$ Ebene durch $\begin{pmatrix}
0\\0\\0
\end{pmatrix}$ und durch $v,w$\\
Ist $w \in \{ av : a \in \R \},$ so ist $E = G$ (aus a))
\item $v,w \ne o\\
G' = \{ w + av : a \in \R\}$\\
$[v \in G' \Leftrightarrow \exists a \in \R: w+ av \in o \Leftrightarrow \exists a \in \R :\: w = (-a)v \in G]$
\end{enumerate}
\subsection{Satz}\label{sec:0.4}
Seien $\mathcal{U}_1,\mathcal{U}_2$ Unterräume von $\R^n$\\
\begin{enumerate}[a)]
\item $\mathcal{U}_1 \cap \mathcal{U}_2$ ist Unterraum von $\R^n$
\item $\mathcal{U}_1 \cup \mathcal{U}_2$ ist im Allgemeinen \textsc{kein} Unterraum von $\R^n$
\item $\mathcal{U}_1 + \mathcal{U}_2 := \{u_1 + u_2 : u_1:\: \mathcal{U}_1, u_2:\: \mathcal{U}_2\}$
(Summe von $\mathcal{U}_1$ und $\mathcal{U}_2$) ist Unterraum von $\R^n$.
\item $\U_1 \subseteq \U_1 + \U_2$\quad$\U_2 \subseteq \U_1+\U_2$ und $\U_1 + \U_2$ ist der kleinste Unterraum von $\R^n$, der $\U_1$ und $\U_2$ enthält. (d.h ist $w$ Unterraum von $\R^n$ mit $\U_1,\U_2 \in w,$ so $\U_1 + \U_2 \subseteq W$)
\end{enumerate}
\begin{proof}
a) $\checkmark$\\
b) TODO\\
c) TODO
\end{proof}
\subsection{Beispiel}\label{sec:0.5}
\begin{enumerate}[a)]
\item \ref{sec:10.3}b)
$G_1 = \{ av :\: a \in \R \}\\
G_2 = \{ aw :\: a \}\\
G_1 + G_2 = E$
\item $\R^3\\
E_1 = \left\{ r \cd \begin{pmatrix}
1\\0\\0
\end{pmatrix} + s \cd \begin{pmatrix}
0\\0\\1
\end{pmatrix}:\: r,s\in\R \right\}\\
\phantom{E_1} = \left\{\begin{pmatrix}
r\\0\\s
\end{pmatrix}:\: r,s\in\R \right\}\\
E_2= \left\{ t \cd \begin{pmatrix}
0\\1\\0
\end{pmatrix} + u \cd \begin{pmatrix}
1\\1\\1
\end{pmatrix} \right\}\\
\phantom{E_2}= \left\{\begin{pmatrix}
u\\t+u\\u
\end{pmatrix} \right\}$\\
$E_1 + E_2$ Unterräume von $\R^3$ (10.3.b)\\
$E_1 \cap E_2 = ?\\
v \in E_1 \cap E_2 \Leftrightarrow v = \begin{pmatrix}
r\\0\\s
\end{pmatrix} = \begin{pmatrix}
u\\t+u\\u
\end{pmatrix} \Leftrightarrow r = u, t + u = 0 , s = u\\
E_1 \cap E_2 = \left\{ \begin{pmatrix}
u\\0\\u 
\end{pmatrix}: u \in \R\right\}\\
\phantom{E_1 \cap E_2} = \left\{ u \cd \begin{pmatrix}
1\\0\\1
\end{pmatrix}: u \in \R \right\}\\
E_1 + E_2 = ?\\
E_1 + E_2 = \R^3$, denn :\\
Es gilt sogar:\\
$\R^3 = E_1 + G_2$, wobei\\
$G_2 = \left\{ t \cd \begin{pmatrix}
0\\1\\0
\end{pmatrix}: t \in \R \right\} \subseteq E_@\\
\begin{pmatrix}
x\\y\\z
\end{pmatrix} = x \cd \begin{pmatrix}
1\\0\\1
\end{pmatrix} z\cd \begin{pmatrix}
0\\0\\1
\end{pmatrix} + y \cd \begin{pmatrix}
0\\1\\0
\end{pmatrix} = \begin{pmatrix}
x\\0\\z
\end{pmatrix} + \begin{pmatrix}
0\\y\\0
\end{pmatrix}\\
\begin{pmatrix}
x\\y\\z
\end{pmatrix} =  (x-y) \begin{pmatrix}
1\\0\\0
\end{pmatrix} + (z-y)\begin{pmatrix}
0\\0\\1
\end{pmatrix} + y \begin{pmatrix}
1\\1\\1
\end{pmatrix}$\\
$\phantom{\begin{pmatrix}
0\\0\\1
\end{pmatrix}}=\begin{pmatrix}
x-y\\0\\z-y
\end{pmatrix} + \begin{pmatrix}
y\\y\\y
\end{pmatrix}$
\end{enumerate}
\subsection{Definition}\label{sec:0.6}
\begin{enumerate}[a)]
\item$v_1,\ldots , v_m \in \R^n, a_1,\ldots a_m \in \R$\\
Dann hei\ss t $a_1v_1 + \ldots + a_m v_m = \sum_{i = 1}^{m} a_iv_i\\
$\emph{Linearkombination} von $v_1,\ldots ,v_m$ (mit Koeffizienten $a_1,\ldots,a_m$).\\
$[ $Zwei formal verschiedene Linearkombinationen der gleichen $v_1,\ldots, v_m$ können den gleichen Vektor darstellen \\
$1 \cd \begin{pmatrix}
1\\0
\end{pmatrix} + 2 \cd \begin{pmatrix}
0\\1
\end{pmatrix} + 3 \cd \begin{pmatrix}
1\\1
\end{pmatrix} = 2 \cd \begin{pmatrix}
1\\0
\end{pmatrix} + 3 \cd \begin{pmatrix}
0\\1
\end{pmatrix} + 2 \cd \begin{pmatrix}
1\\1
\end{pmatrix} = \begin{pmatrix}
4\\5
\end{pmatrix}]$
\item Ist $M \subseteq R^n$, so ist der von M \emph{erzeugte} (oder \emph{aufgespannte}) Unterraum $\langle M \rangle_\R$ (oder $\langle M \rangle$) die Menge aller (endlichen) Linearkombinationen, die man mit Vektoren aus M bilden kann.\\
$\langle M \rangle_\R = \left\{ \sum\limits_{i=1}^{n}a_iv_i : n\in\N , a_i \in\R ,v_i\in M \right\}$ falls $M \ne \varnothing\\
\langle\varnothing\rangle_\R := \{ \varnothing \}\\
M = \{ v_1,\ldots v_m \}$, so TODO...
\end{enumerate}
\subsection{Beispiel}
\begin{enumerate}[a)]
\item $e_i = \begin{pmatrix}
0\\0\\1\\0\\0
\end{pmatrix}\in\R^n\\
\langle e_1,\ldots e_n\rangle = \R^n\\
\begin{pmatrix}
x_1\\\vdots\\x_n
\end{pmatrix} = x_1e_1+x_2e_2+\ldots+x_ne_n$
\item $\U = \langle \begin{pmatrix}
1\\2\\3
\end{pmatrix},\begin{pmatrix}
3\\2\\1
\end{pmatrix},\begin{pmatrix}
2\\3\\4
\end{pmatrix}\rangle_\R$\\
Ist $\U = \R^3$?\\
Für welche $\vektor{x}{y}{z} \in \R^3$ gibt es geeignete Skalare $a,b,c\in\R$ mit $a\vektor{1}{2}{3}+b\vektor{3}{2}{1}+c\vektor{2}{3}{4}?\\$
\[ \begin{matrix}
a &+3b&+2c&=x\\
2a&+2b&+3c&=y\\
3a&+b &+4c&=z
\end{matrix} \]
LGS für die Unbekannten $a,b,c$ mit variabler rechter Seite : Gau\ss\\
\[ \begin{pmatrix}
1&3&2&&x\\
2&2&3&&y\\
3&1&4&&z
\end{pmatrix}\quad\to\quad\begin{pmatrix}
1&3&2&&x\\
2&-4&-1&&y-2x\\
0&-8&-2&&z-3x
\end{pmatrix}   \]\\
\[\to\quad\begin{pmatrix}
1&3&2&&x\\
0&1&\frac14&&\frac{2x-y}4\\
0&0&0&&x-2y+z
\end{pmatrix} \]
LGS ist lösbar $\Leftrightarrow x-2y+z=0$.\\
Dass hei\ss t $\vektor{x}{y}{z} \in \U \Leftrightarrow x -2y = z =0\\
\U = \left\{ \vektor{x}{y}{z} : x-2y+z=0,x,y,z\in\R \right\}\\
\phantom{\U}= \left\{ \vektor{x}{y}{-x+2y} : x,y\in\R \right\}\\
\vektor{2}{3}{4}\in\U$
\end{enumerate}
Lösungen des LGS: $c$ frei wählen, b, a ergeben sich, (falls $x-2y + z = 0)$ z.B $c = 0, b = \frac12x-\frac14y, a = x-3b = -\frac12x + \frac34y$\\
Ist $\mathit{x-2y+z=0}$, so ist\\ $\vektor{x}{y}{z} = (-\frac12x+\frac34y)\vektor{x}{y}{z}+ (\frac12x - \frac14y)\vektor{3}{2}{1}\\
\vektor{2}{3}{4}\frac54\vektor123+\frac14\vektor321\\
\U= \left\langle\vektor123, \vektor321 \right\rangle_\R$
\[\cancel{\begin{matrix}
6x^2&- 3xy&+y^3 &=5\\
7x^3&+ 3x^2y^2&-xy&=7
\end{matrix}} \]
\setcounter{subsection}{8}
\subsection{Definition}\label{sec:0.9}
$v_1,\ldots,v_n \in \R^n$ hei\ss en \emph{linear abhängig}. falls
$a_1,\ldots,a_n \in \R$ existieren, \emph{nicht alle = 0}, mit $a_1v_1 + \ldots +a_nv_n = 0$.\\
Gibt es solche Skalare nicht, so heißen $v_1,\ldots,v_m$ \emph{linear unabhängig} (d.h. aus $a_1v_1\ldots a_nv_n = 0 folgt a_1 = \ldots = a_n = 0$).\\
(Entsprechend $\{ v_1\ldots v_n \}$ linear abhängig/linear unabhängig)\\
Per Definition : $\varnothing$ is linear unabhängig.\\
\subsection{Beispiel}\label{sec:0.10}
\begin{enumerate}[a)]
\item $\sigma$ + v $\in \R^n$ Dann ist v linear unabhängig:\\
Zu zeigen : Ist av = $\sigma \Rightarrow a = 0$\\
Sei $v \vektor{b_1}{\vdots}{b_n}$ Da $v\ne \sigma$,\\
existiert mindestens ein i mit $b_i \ne 0$.\\
Angenommen $\sigma v = \vektor{0 b_1}{\vdots}{0 b_n} = \vektor000 = \sigma $. \\
Dann $ab_i = 0$ Da $b_i \ne 0$, folgt a = 0.\\
$\sigma$ ist linear abhängig: \[1 \cd \sigma = \sigma \]
\item $v_1 = \sigma .v_2\ldots,v_m$ ist linear abhängig :\\
$\sigma = 1 \cd \sigma + 0 \cd v_2 + \ldots + 0 \cd v_m$
\item $v,w \in \R^n\\
v \ne \sigma \ne w$\\
\textcircled{\raisebox{-1pt}{1}}\parbox{.2\textwidth}{v,w sind linear abhängig} $\Leftrightarrow\\
\textcircled{\raisebox{-1pt}{2}} v \in \langle w \rangle_\R \Leftrightarrow\\
\textcircled{\raisebox{-1pt}{3}} w \in \langle v \rangle_\R \Leftrightarrow\\
\textcircled{\raisebox{-1pt}{4}} \langle v \rangle_\R = \langle w \rangle_\R$\\
\textcircled{\raisebox{-1pt}{1}}\\
$v,w$ linear abhängig $\rightarrow \exists a_1,a_2\in \R,$ nicht beide = 0, $a_1v + a_2w = \sigma$. Dann beide $(a_1,a_2) \ne 0\\
a_1v = -a_2w \mid \cd \frac1{a_1}\\
v = - \frac{-a_2}{-a_1}w \in \langle w \rangle_\R  \textcircled{\raisebox{-1pt}{2}}\\
\textcircled{\raisebox{-1pt}{2}}\\
v \in \langle w \rangle_\R$ dass hei\ss t $v = aw$ für ein $a \in \R$ Dann $a \ne 0$, da $v \ne \sigma$. $w = \frac1a \cd v \in \langle v \rangle_\R \textcircled{\raisebox{-1pt}{3}}$\\
\textcircled{\raisebox{-1pt}{3}}\\
$w = bv$ für ein $b \in \R b \ne 0$,da $w \ne \sigma.\\
aw \in \langle w \rangle_\R \Rightarrow aW = (ab)v \in \langle v\rangle_\R\\
\langle w \rangle_\R \subseteq \langle v \rangle_\R$\\
$w = \frac1b w$ Dann analog $\langle v \rangle\R \subseteq \langle w \rangle_\R$\\
Also $ \langle v \rangle\R = \langle w \rangle_\R$\textcircled{\raisebox{-1pt}{4}}\\
\textcircled{\raisebox{-1pt}{4}}\\
$v \in \langle v \rangle_\R = \langle w \rangle_\R$, dass hei\ss t.\\
$ v = a \cd w$ für ein $a \in \R\\
a \cd v + (-a) w = \sigma \Rightarrow v,w$ sind linear abhängig \textcircled{\raisebox{-1pt}{1}}
\item $e_i = \begin{pmatrix}
0\\\vdots\\0\\1\\0\\0\\0
\end{pmatrix} \in \R^n\\
e_1,\ldots,e_n$ sind linear unabhängig.\\
$\sigma = a_1e_1 + \ldots a_ne_n = \begin{pmatrix}
0\\\vdots\\0\\1\\0\\0\\0
\end{pmatrix} + \begin{pmatrix}
0\\\vdots\\0\\a_2\\0\\0\\0
\end{pmatrix} \Rightarrow a_1 = a_2 = \ldots = a_n = 0$
\item $\vektor12{},\vektor{-3}1{},\vektor62{}$ sind linear abhängig $\R^2$: \\
Gesucht sind alle $a_i,b_i \in \R$ mit $a \cd \vektor12{} + b \cd \vektor{-3}1{} + c \cd \vektor62{} = \vektor00{}$\\
Führt auf LGS für a,b,c:\\
$\begin{pmatrix}
1&-3&6&&0\\
2&1&2&&0
\end{pmatrix}\quad\rightarrow\quad \begin{pmatrix}
1&-3&6&&0\\
0&7&-10&&0
\end{pmatrix}\\
c$ ist frei wählbar
\item $\vektor123, \vektor321, \vektor234$ sind linear abhängig in $\R^3$,\\
10.8b) : $\frac54 \vektor123 + \frac14 \vektor321 + (-1)\vektor234 = \vektor000$
\end{enumerate}
\subsection{Satz}\label{sec:0.11}
Seien $v_1,\ldots,v_n \in \R^n$
\begin{enumerate}[a)]
\item $v_1,\ldots,v_m$ sind linear abhängig \textcircled{\raisebox{-1pt}{1}}\\
$\Leftrightarrow \exists i \ldots v_i = \sum\limits^{m}_{\substack{j=1\\j\ne i} }b_jv_j \textcircled{\raisebox{-1pt}{2}} \\\Leftrightarrow \exists i : \langle v_1,\ldots,v_m \rangle_\R = \langle v_1, \ldots v_{i-1}, v_{i+!},\ldots,v_m\rangle_\R$\textcircled{\raisebox{-1pt}{3}}
\item $v_1,\ldots,v_m$ sind linear unabhängig $\Leftrightarrow$ Jedes $v \in \langle v_1,\ldots,v_m\rangle_\R$ lässt sich auf \emph{genau eine} Weise als Linearkombination von $v_1,\ldots,v_m$ schreiben.
\item Sind $v_1,\ldots,v_m$ linear unabhängig und es existiert $v \in \R^n mit v \ne \langle v_1,\ldots,v_m\rangle_\R$ dann sind auch $v_1,\ldots,v_m,v $ linear unabhängig
\end{enumerate}
\begin{proof}
a) \textcircled{\raisebox{-1pt}{1}} $\Rightarrow$ \textcircled{\raisebox{-1pt}{2}}\\
$v_1,\ldots v_m$ sind linear abhängig\\
$\Rightarrow \exists a_1,\ldots,a_m$ nicht alle = 0,\\
$a_av_i + \ldots + a_mv_m = 0$\\
Sei $a_i \ne 0\\
a_iv_i = \sum\limits_{\substack{j =1\\j\ne i}}^{m} -a_jv_j$\\
$\phantom{a_i}v_i = \sum\limits_{\substack{j =1\\j\ne i}}^{m} -\frac{a_j}{a_i}v_j$ \textcircled{\raisebox{-1pt}{2}}\\
\textcircled{\raisebox{-1pt}{2}} $\Rightarrow \textcircled{\raisebox{-1pt}{3}}$\\
Klar : $\langle v_1,\ldots v_{i-1},v_{i+1},v_m \rangle_\R \subseteq \langle v_1, \ldots ,v_m \rangle_\R$\\
Zeige $\supseteq$\quad $v  = \langle v_1,\ldots, v_m \rangle_\R$, d.h\\
$v = \sum\limits_{j=1}^{m} a_jv_j = \sum\limits_{\substack{j=1\\j\ne i}}^{m} a_jv_j + a_i (\sum\limits_{\substack{j=1\\j\ne i}}^{m} b_jv_j) = \sum\limits_{\substack{j=1\\j\ne i}}^{m} (a_j + a_i b_j)v_j \in \langle v_1,\ldots v_{i-1},v_{i+1}.\ldots,v_m \rangle_\R \textcircled{\raisebox{-1pt}{2}}$
\textcircled{\raisebox{-1pt}{3}} $ \Rightarrow \textcircled{\raisebox{-1pt}{1}}\\
v_i \in \langle v_1 \ldots v_m \rangle_\R = \langle v_1 \ldots v_{i-1}, v_{i+1}, \rangle_\R$, dass hei\ss t es existiert \\
$a_1, \ldots a_{i-1},a_{i+1},\ldots a_{m} \in \R$ mit \[ v_i = \sum_{\substack{j=1\\j \ne i}}^{m} a_jv_j \]
$\Rightarrow \sigma = a_1 + v_1 + \ldots + a_{i-1}v_{i-1} + (-1)v_i + a_{i+1}v_{i+1} + \ldots a_mv_m \qquad v_1\ldots v_m $ linear abhängig 
%TODO Beweis b und c
\end{proof}
\subsection{Satz}\label{sec:0.12}
Sind $v_i, \ldots, v_{n+1} \in \R^n$, so\\
sin $v_i,\ldots, v_{n+1}$ linear abhängig.\\
(Insbesondere ist $m > n$ und $v_i,v_m \in \R^n$, so sind $v_1,\ldots,v_m $ linear abhängig)
\begin {proof}
Suche alle $a_1,\ldots,a{n+1} \in \R$ mit $a_iv_1 + \ldots a_{n+1}v_{n+1} = \vektor0\dots0$\\
Führt zu LGS für $a_1,\ldots, a_{n+1}$ mit Koeffizientenmatrix $(v_1,\ldots, v_2, \ldots, v_{n+1}) =A$\\
Frage : Hat $ A \cd \vektor{a_i}\vdots{a_{n+1}} = \vektor0\vdots0 \in \R^n$ nicht triviale Lösung?\\
Gau\ss\ :\\
$\left(\raisebox{-0.5cm}{\scalebox{4}{A}} \vektor0\vdots0 \longrightarrow \right)$
%TODO Rest von 10.12
\end {proof}
\subsection{Definition}\label{sec:0.13}
Sei $\U$ ein Unterraum von $\R^n$\\
$B \subseteq \U$ hei\ss t Basis von $\U$ falls:
\begin{enumerate}[(1)]
\item $\langle B \rangle_\R = U$
\item B ist linear unabhängig 
\end{enumerate}
($\U = \{\sigma\}, B = \varnothing$)
\subsection{Beispiel}\label{sec:0.14}
\begin{enumerate}[a)]
\item $e_1,\ldots,e_n$ ist Basis von $\R^n$ (kanonische Basis)\\
$e_1 = \begin{pmatrix}
0\\\vdots\\1\\0\\0
\end{pmatrix}\leftarrow i\\
\vektor{a_i}\vdots{a_n} = \sum_{i=1}^{n} a_ie_i$
\item $\vektor12{},\vektor32{}$ ist Basis von $R^2$:\\
Sei $\vektor{x}{y}{}\in R^2$. Gesucht: $a,b \in \R$ mit $a\vektor12{}+b\vektor32{} =\vektor{x}{y}{}$\\
LGS mit variabler rechter Seite\[
\begin{matrix}
1a &+3b&=x\\
2a &+2b&=y
\end{matrix}\]
Gau\ss:\medskip\\
$\begin{pmatrix}
1&3&&x\\
2&2&&y
\end{pmatrix}\quad\rightarrow\quad\begin{pmatrix}
1&3&&x\\
0&-4&&y-2x
\end{pmatrix}$\\
Eindeutige Lösung: $b = -\frac14y+\frac12x\quad a = x-3b = x +\frac34y - \frac32x = -\frac12x + \frac34y$\\
z.B $\vektor{x}{y}{} = \vektor10{}\\
\vektor10{} = -\frac12\vektor12{}+ \frac12\vektor32{}
\R^2 \langle \vektor12{},\vektor32{}\rangle\\
\vektor12{},\vektor32{}$ sind linear unabhängig nach \ref{sec:0.10}c)\\
$\left\{\vektor12{},\vektor32{}\right\}$ Basis.
\item $\U = \left\langle \vektor123,\vektor321,\vektor234\right\rangle_\R\\
\vektor234 = \frac54 \vektor123 + \frac14\vektor321\\
\U = \left\langle \vektor123,\vektor321\right\rangle_\R\\
\vektor123,\vektor321$ linear unabhängig (\ref{sec:0.10}c))\\
$\left\{ \vektor123, \vektor321 \right\}$ Basis von $\U$
\end{enumerate}
\subsection{Satz}\label{sec:0.15}
Jeder Unterraum $\U$ des $\R^n$ besitzt eine Basis.
\begin{proof}
Ist $\U = \{\sigma\}$,so $b = \varnothing$.\\
Sei also $\U \ne \{\sigma\}$.\\
$v_1$ ist linear unabhängig.\\
$\langle v_1\rangle_\R \subseteq \U.$\\
Ist $\U = \langle v_1 \rangle_\R$, so ist $\{ v_1 \}$ Basis von $\U$\\
Ist $\langle v_1 \rangle_\R \subsetneq \U$.\\
Sei $v_2 \in \U \setminus \langle v_1 \rangle_\R$.\\
Nach \ref{sec:0.11}c) ist $\{ v_1,v_2 \}$ linear unabhängig. Ist $\rangle v_1,v_2 \rangle = \U$, so ist $\{ v_1.v_2 \}$ Basis von $\U$.\\
Ist $\langle v_1, v_2 \rangle_\R \subsetneq U$ so wähle $v_3$ usw.\\
Es existiert $m \ne n$ mit $\langle v_1,\ldots v_m\rangle_\R = \U$ und $v_1\ldots,v_m$ sind linear unabhängig.\\
(Denn noch \ref{sec:0.12} gibt es im $\R^n$ keine n+1 linear unabhängige Vektoren)
\end{proof}
\subsection{Satz}\label{sec:0.16}
Je zwei Basen $B_1,B_2$ eines Unterraums $\U$ des $\R^n$ enthalten die gleiche Anzahl von Vektoren $\abs{B_1} = \abs{B_2}$.\\
Insbesondere:\\
Je zwei Basen des $\R^n$ enthalten n Vektoren
\subsection{Definition}\label{sec:0.17}
Ist $\U$ Unterraum von $\R^n$, B Basis von $\U,\, \abs{B} = m$.\\
Dann ist $m$ die \emph{Dimension} von $\U$, $\dim(u) = m.$\\
$\dim(\R^n) = n,\, \dim(\U) \ne n$.
\subsection{Satz (Basisergänzungssatz)}\label{sec:0.18}
Sei $\U$ Unterraum der $\R^n, M \subseteq \U$ eine Menge m linear unabhängiger Vektoren. Dann lässt sich M zu einer Basis von $\U$ ergänzen.
\begin{proof}
Analog zu \ref{sec:0.15}
\end{proof}
\subsection{Korollar}
Ist $\U$ Unterraum des $\R^n$ und $\dim(\U) = n$, dann ist $\U = \R^n$
\begin{proof}
Sei B Basis von $\U$, also $\abs{B} = n$.\\
Nach \ref{sec:0.18} (dort mit $\U = \R^n, M = B$) lässt sich B zu Basis B' von $\R^n$ ergänzen.\\
$\dim(\R^n) = n \Rightarrow \abs{B'} = n.$\\
Also B = B'\\
$\R^n = \langle B'\rangle_\R = \langle B \rangle_\R = \U$
\end{proof}
\subsection{Definition}
Ist $\U$ Unterraum von $\R^n$, B = ($u_1\ldots,u_m$) eine geordnete Basis von $\U$. Nach \ref{sec:0.11}b), lässt sich jeder Vektorraum $\U = \langle B\rangle_\R$ \emph{eindeutig} als Linearkombination \[ \U = \sum_{i=1}^{m} a_iu_i\quad,a_i \in \R \] schreiben.\\
$(a_1\ldots,a_m)$ hei\ss en \emph{Koordinaten} von u bzgl. der Basis B.
\subsection{Beispiele}
\begin{enumerate}[a)]
\item B($e_1\ldots,e_m$) kanonische Basis von $\R^n$.\\
Koordinaten von $\vektor{a_1}\vdots{a_n} \in \R^n$ bzgl. B:\\
$(a_1\ldots,a_n)$ \emph{kartesische} Koordinaten. \hfill (\emph{Rene Descartes, 1596-1650}) 
\end{enumerate}
\newpage
\begin{center}
\Huge Anfang des WS 2015/16
\end{center}
\section{Algebraische Strukturen}
\subsection{Definition}\label{sec:1.1}
Sei $X \neq \varnothing$. Eine \index{Verknüpfung}\emph{Verknüpfung} auf $X$ ist \index{Abbildung}: \\ $\begin{cases}
X \times X &\longrightarrow X\\
(a,b) &\longrightarrow a \star b
\end{cases}$ ('Produkt' von a und b)\\
$\star$ ist Platzhalter für andere \index{Verknüpfungssymbole}Verknüpfungssymbole, die in speziellen Beispielen auftreten können.
\subsection{Beispiele}\label{sec:1.2}
\begin{enumerate}[a)]
\item Addition $+$ und Multiplikation $\cd$ sind Verknüpfungen auf $\N, \Z, \mathbb{Q}, \R, \mathbb{C}$
Multiplikation ist \emph{keine} Verknüpfung auf der Menge der negativen ganzen Zahlen.
\item Division ist keine Verknüpfung auf $\N$. Division ist Verknüpfung auf $\mathbb{Q} \setminus \{Q\}, \R \setminus \{0\}, \C \setminus \{0\}$
\item $\Z_n : = \{ 0,1,\ldots,n-1 \}$ \hfill $(n \in \N)\\
a \oplus b : = (a + b) \mod n \in \Z_n\\
a \odot b := (a \cd b) \mod n \in \Z_n$\\
Verknüpfungen auf $\Z_n\\
n = 7:\qquad 5 \odot 6 = 2\\
\phantom{n = 7::}\qquad 5 \oplus 6 = 4\\
n = 2:\qquad \Z_n = \{0,1\}\\
\phantom{n = 2::}\qquad0 \oplus 0 = 0, 1 \oplus 0 = 1,0 \oplus 1 = 1, 1 \oplus 1 = 0\\
\phantom{n = 2::}\qquad\odot = \cdot$
\item M Menge, X = Menge aller Abbildungen $M \longrightarrow M$. Verknüpfung auf X: Hintereinanderausführung von Abbildungen: $\circ$\\
$(f, g): M \longrightarrow M$, So $f\circ g:\:M \to M\\
(f \circ g)(m) = f(g(m)) \in M, m \in M$\\
Im Allgemeinen ist $g \circ f \ne f \circ g$
\item $X = \{0,1\}$\\
2-stellige Aussagen, Junktoren wie $\land,\lor,$\textsf{ XOR }$,\Rightarrow,\Leftrightarrow$ hei\ss en Verknüpfungen auf $X$.
0 entspricht f, 1 entspricht w\\
$0 \lor 0 = 0, 1 \lor 0 = 1, 0 \lor 1 = 1, 1 \lor 1 = 1\\
0 \land 0 = 0, 0 \land 1 = 0, 1 \land 0 = 0, 1 \land 1 =1$ (= 'Multiplikation')\\
$0 \textsf{ XOR } 0 = 0, 1 \textsf{ XOR } 0 = 1, 0 \textsf{ XOR } 1 = 1, 1 \textsf{ XOR } 1 = 0$ (= Addition mod 2)
\item $X = M_n(\R)$ = Menge der  $n \times n$- Matrizen über $\R$.\\ \index{Matrizenaddition}Matrizenaddition ist Verknüpfung auf $X$\\
\index{Matrizenmultiplikation}Matrizenmultiplikation ist Verknüpfung auf $X$.
\item $M$ Menge, $X$, Menge aller endlichen Folgen von Elementen aus M ('Wörter' über M)\\
Verknüpfung: Hintereinanderausführung zweier Folgen (\index{Konkatenation}Konkatenation)\\
$M = \{0,1\} w_1 = 1101 w_2 = 001\\
w_1w_2 = 110111\\
w_2w_1 = 0011101$\\
\end{enumerate}
\subsection{Definition}\label{sec:1.3}
Sei $X \ne 0$ eine Menge mit Verknüpfung $\star$.
\begin{enumerate}[a)]
\item $X$, genauer $(X,\star)$ ist \index{Halbgruppe}\emph{Halbgruppe}, falls
($a \star b) \star c = a \star (b \star c)$ für alle $a,b,c \in X.$ (\index{\emph{Assoziativgesetz}})
\item $(X,\star)$ hei\ss t \index{Monoid}\emph{Monoid}, falls $(X,\star)$ Halbgruppe ist und ein $e \in X$ existiert mit $e \star a = a$ und $a \star e = a$ für alle $a \in X$. e hei\ss t \index{neutrales Element}\emph{neutrales Element} (später, e ist eindeutig bestimmt)
\item Sei $(X,\star)$ ein Monoid. Ein Element $a \in X$ hei\ss t \index{invertierbar}\emph{invertierbar}, falls $b \in X$ existiert (abhängig von a) mit $a \star b = b \star a = e$. b hei\ss t \emph{inverses Element}\index{inverses Element} (das \emph{Inverse}\index{Inverse}) zu a. (später: wenn b existiert, so ist es eindeutig bestimmt)
\item Monoid $(X,\star)$ hei\ss t \index{Gruppe}\emph{Gruppe}, falls jedes Element in $X$ bezüglich $\star$ invertierbar ist.
\item Halbgruppe, Monoid, Gruppe $(X,\star)$ bezüglich kommutativ (oder \index{abelsch}\emph{abelsch}) falls $a \star b = b \star a $ für alle $a,b \in X$ (Kommutativgesetz) \\(Nach: Abel, 1802-1829)
\end{enumerate}
\printindex
\end{document}