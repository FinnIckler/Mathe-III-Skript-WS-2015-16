\documentclass[a4paper,12pt,twoside]{article}
\usepackage{fourier}
\usepackage{polynom}
\usepackage[ngerman]{babel}
\usepackage[leqno,tbtags,nointlimits]{amsmath}
\usepackage{amssymb,amsthm,amsfonts}
\usepackage{graphicx}
\usepackage{ifthen}
%\usepackage{gauss}
\usepackage{tikz}
\usepackage{mathtools}
\usepackage{makeidx}
\usepackage{fancyhdr,lastpage}
\usepackage{enumerate}
\usepackage[onehalfspacing]{setspace}
\usepackage{mdsymbol}
\usepackage{marvosym}
\usepackage{cancel}
\usepackage{pgfplots}
\usepackage{color}
\usepackage{bigints}
\usepackage{array}
\usepackage{mdframed}
\usepackage{marginnote}
\usetikzlibrary{trees,automata,arrows,shapes,decorations.pathmorphing,matrix}
\pagestyle{fancy}
\usepackage{hyperref}
\fancyhf{} %--Clear all fields
\renewcommand\sectionmark[1]{ \markboth{\thesection\ \textsc{#1}}{}}
\fancyhead[LO,RE]{\small \leftmark}
\fancyhead[LE,RO]{ \rightmark}
\fancyfoot{} % clear all footer fields
\fancyfoot[LE,RO]{\thepage}
\newcommand{\cd}{\cdot}
\newcommand{\C}{\mathbb{C}}
\newcommand{\Z}{\mathbb{Z}}
\renewcommand{\i}{\item}
\newcommand{\U}{\mathcal{U}}
\newcommand{\N}{\mathbb{N}}
\newcommand{\R}{\mathbb{R}}
\newcommand{\raum}[1]{\left\langle#1\right\rangle}
\DeclarePairedDelimiter{\ceil}{\lceil}{\rceil}
\DeclarePairedDelimiter{\floor}{\lfloor}{\rfloor}
\usepackage[normalem]{ulem}
\usepackage{blkarray}
\usepackage{stmaryrd}
\usepackage{titletoc}
\newcommand{\abs}[1]{\lvert #1 \rvert}
\renewcommand\headrule{{\color{gray}%
\hrule height 2pt width\headwidth
\vspace{1pt}%
\hrule height 1pt width\headwidth
\vspace{-4pt}}}
\makeatletter
\newcommand{\resetHeadWidth}{\fancy@setoffs}
\makeatother
\newcommand{\limit}[1]{\displaystyle \lim_{#1}}
\usepgfplotslibrary{fillbetween}
\pgfplotsset{compat=1.9}
\newcommand\vektor[3]{\begin{pmatrix}
#1\\#2\\#3
\end{pmatrix}}
\newcommand\vektort[2]{\begin{pmatrix}
#1\\#2
\end{pmatrix}}
\newcommand\vektorf[4]{\begin{pmatrix}
#1\\#2\\#3\\#4
\end{pmatrix}}
\newcount\vectorcount
\renewcommand*\vector[1]{%
  \global\vectorcount#1
  \begin{pmatrix}
    \vectornext
}
\def\vectornext#1{%
  #1
  \global\advance\vectorcount-1
    \ifnum\vectorcount>0
      \\
      \expandafter\vectornext
    \else
      \end{pmatrix}
    \fi
}
\DeclareMathOperator{\Grad}{Grad}
\makeindex
\hypersetup{%
pdfborder = {0 0 0}
}
\begin{document}
\resetHeadWidth
\newpage
\tableofcontents
\listoffigures
\clearpage
\setcounter{section}{-1}
\begin{center}
\Huge Ende des SS 2015
\end{center}
\section{Der Vektorraum $\R^n$}\index{Vektorraum}
$n \in \N\quad \R^n = \left\{ \begin{pmatrix}
a_1\\
\vdots\\
a_n
\end{pmatrix} : a_1 \in \R \right\}$\\
\emph{Spaltenvektoren}\index{Spaltenvektoren} der Länge $n:\: \begin{pmatrix}
a_1\\\vdots\\a_n
\end{pmatrix} = (a_1,\ldots,a_n)^t\\
a_1,\ldots,a_n$ \emph{Komponente}\index{Komponente} der Spaltenvektoren.\\
Wie bei Matrizen:\\
\begin{minipage}{.5\textwidth}\[ \begin{pmatrix}
a_1\\\vdots\\a_n
\end{pmatrix}+ \begin{pmatrix}
b_1\\\vdots\\b_n
\end{pmatrix} = \begin{pmatrix}
a_1 + b_1\\\vdots\\a_n + b_n
\end{pmatrix}\]
\end{minipage}%
\begin{minipage}{.5\textwidth}
(Multiplikation entspricht der Matrizenmultiplikation\index{Matrizenmultiplikation} und ist nicht möglich falls $n > 1$)
\end{minipage}\\
Multiplikation eines Spaltenvektors mit einer Zahl (\emph{Skalar})
\[ a \cd \begin{pmatrix}
a_1\\\vdots\\a_n
\end{pmatrix} = \begin{pmatrix}
aa_1\\\vdots\\aa_n
\end{pmatrix} \]
Addition+Abbildung : $\R^n \times \R^n \to \R^n$\\
$\R^n$ mit Addition und Multiplikation mit Skalaren : \emph{$\R$-Vektorraum}\\
Die Vektoren im $\R^1 (= \R),\R^2$ und $\R^3$ entsprechen Punkten auf der \index{Zahlengerade}Zahlengerade, Ebene, dreidimensionalen Raums.
\begin{figure}[h!]
\centering
\caption{Ein \index{Vektor}Vektor dargestellt durch seinen Ortsvektor}
\begin{tikzpicture}
\begin{axis}[axis equal, axis y line = center, axis x line = center, xmax = 3 , ymax = 3, ymin = 0 , xmin = -0.5,xtick={0.1,2},ytick={2},xticklabels={$\begin{pmatrix}
0\\0
\end{pmatrix}$,$a_1$},yticklabels={$a_2$}]
\draw[->] (axis cs:0,0)--(axis cs: 2,2);
\end{axis}
\end{tikzpicture}
\end{figure}
Punkte des $\R^2,\R^3$ lassen sich identifizieren mit, {\em Ortsvektoren}\index{Ortsvektoren} Pfeile mit Beginn in 0 (Komp = 0) und Ende im entsprechenden Punkt\\
Addition von Spaltenvektoren entspricht der Addition von Ortsvektoren entsprechend der \index{Parallelogrammregel}Parallelogrammregel.
\begin{figure}[h!]
\centering
\caption{Vektoraddition durch Parallelogrammbildung}
\begin{tikzpicture}
\begin{axis}[
axis x line=center,
axis y line=center,
axis equal,
ymin = -1,
xmin = -3,
ymax = 7,
xmax = 5.5,
xtick ={4},
xticklabels={a},
ytick ={1},
yticklabels={b}]
\addplot [black, mark = *, nodes near coords=,every node near coord/.style={anchor=180}] coordinates {( 4, 3)};
   \draw[->](axis cs:0,0)--(axis cs:3.88,2.88);
       \addplot [black, mark = *, nodes near coords=,every node near coord/.style={anchor=0}] coordinates {( -1, 1)};
              \draw[->](axis cs:0,0)--(axis cs:-0.88,0.88);
\addplot [black, mark = *, nodes near coords=,every node near coord/.style={anchor=0}] coordinates {( 3, 4)};
\addplot[mark=none, black] coordinates {(-1,1) (3,4)};
\addplot[mark=none, black] coordinates {(3,4) (4,3)};
\draw[-> ,red](axis cs:0,0)--(axis cs:2.95,3.88);
\end{axis}
\end{tikzpicture}\end{figure}
Multiplikation mit Skalaren a :\\ Streckung (falls $\abs{a} > 1$)\\ Stauchung (falls $0 \ge \abs{a} \ge 1$)\\
Richtungspunkt, falls $a < 0$
%TODO: Steckung und Stauchung\\
\subsection{Satz (Rechenregeln in $\R^n$)}\label{sec:0.1}
Seien $u,v,w \in \R^n,a,b\in\R$ Dann gilt:\\
\begin{enumerate}[a)]
\item \begin{align}
u + (v + w) &= (u + v) + w \tag{1.1}\\
v + 0 = 0 + v &= v, \text{wobei } 0\ \textit{Nullvektor} \tag{1.2}\\
v + -v &= 0 \tag{1.3}\marginnote{$\R^n$ kommutative Gruppe}\\
u + v &= v + u \tag{1.4}\\
(a + b)v &= av + bv \tag{2.1}\\
a(u + v) &= au + av \tag{2.2}\\
(a \cd b)v &=a(bv)\tag{2.3}\\
1v &= v \tag{2.4}
\end{align}
\item $0 \cdot v = 0$ und $ a \cdot 0 = 0$\\
Beweis folgt aus entsprechenden Rechenregeln in 0
\end{enumerate}
\subsection{Definition}\label{sec:0.2}
Eine nicht-leere Teilmenge $\mathcal{U} \supset \R^n$ hei\ss t \emph{Unterraum} (oder \emph{Teilraum} von $\R^n$), falls gilt:
\begin{enumerate}[(1)]
\item $\forall u_1,\,u_2 \in \mathcal{U}:\: u_1 + u_2 \in \mathcal{U}$ (Abgeschlossenheit bezüglich +)
\item $\forall u \in \mathcal{U} \forall a \in \R:\: au \in \mathcal{U}
$(Abgeschlossenheit bezüglich Mult. mit Skalaren)
\end{enumerate}
$\mathcal{U}$ enthält Nullvektor \{0\} Unterraum\index{Unterraum} von $\R^n$ (Nullraum)\index{Nullraum}\\
$\R^n $ ist Unterraum von $\R$
\subsection{Beispiele}\label{sec:0.3}
\begin{enumerate}[a)]
\item $0 \ne v \in \R^2\quad G = \{ av: a \in \R \}$ ist Unterraum von $\R^n$\\\begin{minipage}{.3\textwidth}
($a_1v,a_2v \in G, (a_1 + a_2)v \in G$\quad2.1 in \ref{sec:0.2}\\
$av \in G, b \in \R (ba)v \in G$)
\end{minipage}\\
G = Ursprungsgerade durch $\begin{pmatrix}
0\\0
\end{pmatrix}$ und v = $\begin{pmatrix}
a_1\\a_2
\end{pmatrix}
n = 2$:
\begin{figure}[h!]
\centering
\caption{Gerade dargestellt durch Vektoren}
\begin{tikzpicture}
\begin{axis}[axis equal, axis y line = center, axis x line = center, xmax = 3 , ymax = 2, ymin = -2, xmin = -0.5,xtick={0.1,2},ytick={2},xticklabels={$\begin{pmatrix}
0\\0
\end{pmatrix}$,$a_1$},yticklabels={$a_2$}]
\draw[blue,thick] (axis cs:-2,-2)--(axis cs: 2,2);
\end{axis}
\end{tikzpicture}
\end{figure}
\item $v,w \in \R^n\\
E = \{ av + bw : a,b \in \R \}$ ist Unterraum von $\R^n\\
v = o, w = o :\: E = \{ o \}\\
v \ne o\quad w \not\in \{ av:\: a \in \R \}\\
E = \R^2 $
$n = 3:\:$ Ebene durch $\begin{pmatrix}
0\\0\\0
\end{pmatrix}$ und durch $v,w$\\
Ist $w \in \{ av : a \in \R \},$ so ist $E = G$ (aus a))
\item $v,w \ne o\\
G' = \{ w + av : a \in \R\}$\\
$[v \in G' \Leftrightarrow \exists a \in \R: w+ av \in o \Leftrightarrow \exists a \in \R :\: w = (-a)v \in G]$
\end{enumerate}
\subsection{Satz}\label{sec:0.4}
Seien $\mathcal{U}_1,\mathcal{U}_2$ Unterräume von $\R^n$
\begin{enumerate}[a)]
\item $\mathcal{U}_1 \cap \mathcal{U}_2$ ist Unterraum von $\R^n$
\item $\mathcal{U}_1 \cup \mathcal{U}_2$ ist im Allgemeinen \textsc{kein} Unterraum von $\R^n$
\item $\mathcal{U}_1 + \mathcal{U}_2 := \{u_1 + u_2 : u_1:\: \mathcal{U}_1, u_2:\: \mathcal{U}_2\}$
(Summe von $\mathcal{U}_1$ und $\mathcal{U}_2$) ist Unterraum von $\R^n$.
\item $\U_1 \subseteq \U_1 + \U_2$\quad$\U_2 \subseteq \U_1+\U_2$ und $\U_1 + \U_2$ ist der kleinste Unterraum von $\R^n$, der $\U_1$ und $\U_2$ enthält. (d.h ist $w$ Unterraum von $\R^n$ mit $\U_1,\U_2 \in w,$ so $\U_1 + \U_2 \subseteq W$)
\end{enumerate}
\begin{proof}
a) $\checkmark$\\
b) %TODO\\
c) %TODO
\end{proof}
\subsection{Beispiel}\label{sec:0.5}
\begin{enumerate}[a)]
\item \ref{sec:10.3}b)
$G_1 = \{ av :\: a \in \R \}\\
G_2 = \{ aw :\: a \}\\
G_1 + G_2 = E$
\item $\R^3\\
E_1 = \left\{ r \cd \begin{pmatrix}
1\\0\\0
\end{pmatrix} + s \cd \begin{pmatrix}
0\\0\\1
\end{pmatrix}:\: r,s\in\R \right\}\\
\phantom{E_1} = \left\{\begin{pmatrix}
r\\0\\s
\end{pmatrix}:\: r,s\in\R \right\}\\
E_2= \left\{ t \cd \begin{pmatrix}
0\\1\\0
\end{pmatrix} + u \cd \begin{pmatrix}
1\\1\\1
\end{pmatrix} \right\}\\
\phantom{E_2}= \left\{\begin{pmatrix}
u\\t+u\\u
\end{pmatrix} \right\}$\\
$E_1 + E_2$ Unterräume von $\R^3$ (10.3.b)\\
$E_1 \cap E_2 = ?\\
v \in E_1 \cap E_2 \Leftrightarrow v = \begin{pmatrix}
r\\0\\s
\end{pmatrix} = \begin{pmatrix}
u\\t+u\\u
\end{pmatrix} \Leftrightarrow r = u, t + u = 0 , s = u\\
E_1 \cap E_2 = \left\{ \begin{pmatrix}
u\\0\\u 
\end{pmatrix}: u \in \R\right\}\\
\phantom{E_1 \cap E_2} = \left\{ u \cd \begin{pmatrix}
1\\0\\1
\end{pmatrix}: u \in \R \right\}\\
E_1 + E_2 = ?\\
E_1 + E_2 = \R^3$, denn :\\
Es gilt sogar:\\
$\R^3 = E_1 + G_2$, wobei\\
$G_2 = \left\{ t \cd \begin{pmatrix}
0\\1\\0
\end{pmatrix}: t \in \R \right\} \subseteq E_@\\
\begin{pmatrix}
x\\y\\z
\end{pmatrix} = x \cd \begin{pmatrix}
1\\0\\1
\end{pmatrix} z\cd \begin{pmatrix}
0\\0\\1
\end{pmatrix} + y \cd \begin{pmatrix}
0\\1\\0
\end{pmatrix} = \begin{pmatrix}
x\\0\\z
\end{pmatrix} + \begin{pmatrix}
0\\y\\0
\end{pmatrix}\\
\begin{pmatrix}
x\\y\\z
\end{pmatrix} =  (x-y) \begin{pmatrix}
1\\0\\0
\end{pmatrix} + (z-y)\begin{pmatrix}
0\\0\\1
\end{pmatrix} + y \begin{pmatrix}
1\\1\\1
\end{pmatrix}$\\
$\phantom{\begin{pmatrix}
0\\0\\1
\end{pmatrix}}=\begin{pmatrix}
x-y\\0\\z-y
\end{pmatrix} + \begin{pmatrix}
y\\y\\y
\end{pmatrix}$
\end{enumerate}
\subsection{Definition}\label{sec:0.6}
\begin{enumerate}[a)]
\item$v_1,\ldots , v_m \in \R^n, a_1,\ldots a_m \in \R$\\
Dann hei\ss t $a_1v_1 + \ldots + a_m v_m = \sum_{i = 1}^{m} a_iv_i\\
$\emph{Linearkombination}\index{Linearkombination} von $v_1,\ldots ,v_m$ (mit Koeffizienten $a_1,\ldots,a_m$).\\
$[ $Zwei formal verschiedene Linearkombinationen der gleichen $v_1,\ldots, v_m$ können den gleichen Vektor darstellen \\
$1 \cd \begin{pmatrix}
1\\0
\end{pmatrix} + 2 \cd \begin{pmatrix}
0\\1
\end{pmatrix} + 3 \cd \begin{pmatrix}
1\\1
\end{pmatrix} = 2 \cd \begin{pmatrix}
1\\0
\end{pmatrix} + 3 \cd \begin{pmatrix}
0\\1
\end{pmatrix} + 2 \cd \begin{pmatrix}
1\\1
\end{pmatrix} = \begin{pmatrix}
4\\5
\end{pmatrix}]$
\item Ist $M \subseteq R^n$, so ist der von M \emph{erzeugte} (oder \emph{aufgespannte}) Unterraum $\langle M \rangle_\R$ (oder $\langle M \rangle$) die Menge aller (endlichen) Linearkombinationen, die man mit Vektoren aus M bilden kann.\\
$\langle M \rangle_\R = \left\{ \sum\limits_{i=1}^{n}a_iv_i : n\in\N , a_i \in\R ,v_i\in M \right\}$ falls $M \ne \varnothing\\
\langle\varnothing\rangle_\R := \{ \varnothing \}\\
M = \{ v_1,\ldots v_m \}$, so %TODO...
\end{enumerate}
\subsection{Beispiel}\label{sec:10.7}
\begin{enumerate}[a)]
\item $e_i = \begin{pmatrix}
0\\0\\1\\0\\0
\end{pmatrix}\in\R^n\\
\langle e_1,\ldots e_n\rangle = \R^n\\
\begin{pmatrix}
x_1\\\vdots\\x_n
\end{pmatrix} = x_1e_1+x_2e_2+\ldots+x_ne_n$
\item $\U = \langle \begin{pmatrix}
1\\2\\3
\end{pmatrix},\begin{pmatrix}
3\\2\\1
\end{pmatrix},\begin{pmatrix}
2\\3\\4
\end{pmatrix}\rangle_\R$\\
Ist $\U = \R^3$?\\
Für welche $\vektor{x}{y}{z} \in \R^3$ gibt es geeignete Skalare $a,b,c\in\R$ mit $a\vektor{1}{2}{3}+b\vektor{3}{2}{1}+c\vektor{2}{3}{4}?\\$
\[ \begin{matrix}
a &+3b&+2c&=x\\
2a&+2b&+3c&=y\\
3a&+b &+4c&=z
\end{matrix} \]
LGS für die Unbekannten $a,b,c$ mit variabler rechter Seite : Gau\ss\\
\[ \begin{pmatrix}
1&3&2&&x\\
2&2&3&&y\\
3&1&4&&z
\end{pmatrix}\quad\to\quad\begin{pmatrix}
1&3&2&&x\\
2&-4&-1&&y-2x\\
0&-8&-2&&z-3x
\end{pmatrix}   \]\\
\[\to\quad\begin{pmatrix}
1&3&2&&x\\
0&1&\frac14&&\frac{2x-y}4\\
0&0&0&&x-2y+z
\end{pmatrix} \]
LGS ist lösbar $\Leftrightarrow x-2y+z=0$.\\
Dass hei\ss t $\vektor{x}{y}{z} \in \U \Leftrightarrow x -2y = z =0\\
\U = \left\{ \vektor{x}{y}{z} : x-2y+z=0,x,y,z\in\R \right\}\\
\phantom{\U}= \left\{ \vektor{x}{y}{-x+2y} : x,y\in\R \right\}\\
\vektor{2}{3}{4}\in\U$
\end{enumerate}
Lösungen des LGS: $c$ frei wählen, b, a ergeben sich, (falls $x-2y + z = 0)$ z.B $c = 0, b = \frac12x-\frac14y, a = x-3b = -\frac12x + \frac34y$\\
Ist $\mathit{x-2y+z=0}$, so ist\\ $\vektor{x}{y}{z} = (-\frac12x+\frac34y)\vektor{x}{y}{z}+ (\frac12x - \frac14y)\vektor{3}{2}{1}\\
\vektor{2}{3}{4}\frac54\vektor123+\frac14\vektor321\\
\U= \left\langle\vektor123, \vektor321 \right\rangle_\R$
\[\cancel{\begin{matrix}
6x^2&- 3xy&+y^3 &=5\\
7x^3&+ 3x^2y^2&-xy&=7
\end{matrix}} \]
\setcounter{subsection}{8}
\subsection{Definition}\label{sec:0.9}
$v_1,\ldots,v_n \in \R^n$ hei\ss en \emph{linear abhängig}. falls
$a_1,\ldots,a_n \in \R$ existieren, \emph{nicht alle = 0}, mit $a_1v_1 + \ldots +a_nv_n = 0$.\\
Gibt es solche Skalare nicht, so heißen $v_1,\ldots,v_m$ \emph{linear unabhängig} (d.h. aus $a_1v_1\ldots a_nv_n = 0 folgt a_1 = \ldots = a_n = 0$).\\
(Entsprechend $\{ v_1\ldots v_n \}$ linear abhängig/linear unabhängig)\\
Per Definition : $\varnothing$ is linear unabhängig.\\
\subsection{Beispiel}\label{sec:0.10}
\begin{enumerate}[a)]
\item $\sigma$ + v $\in \R^n$ Dann ist v linear unabhängig:\\
Zu zeigen : Ist av = $\sigma \Rightarrow a = 0$\\
Sei $v \vektor{b_1}{\vdots}{b_n}$ Da $v\ne \sigma$,\\
existiert mindestens ein i mit $b_i \ne 0$.\\
Angenommen $\sigma v = \vektor{0 b_1}{\vdots}{0 b_n} = \vektor000 = \sigma $. \\
Dann $ab_i = 0$ Da $b_i \ne 0$, folgt a = 0.\\
$\sigma$ ist linear abhängig: \[1 \cd \sigma = \sigma \]
\item $v_1 = \sigma .v_2\ldots,v_m$ ist linear abhängig :\\
$\sigma = 1 \cd \sigma + 0 \cd v_2 + \ldots + 0 \cd v_m$
\item $v,w \in \R^n\\
v \ne \sigma \ne w$\\
\textcircled{\raisebox{-1pt}{1}}\parbox{.2\textwidth}{v,w sind linear abhängig} $\Leftrightarrow\\
\textcircled{\raisebox{-1pt}{2}} v \in \langle w \rangle_\R \Leftrightarrow\\
\textcircled{\raisebox{-1pt}{3}} w \in \langle v \rangle_\R \Leftrightarrow\\
\textcircled{\raisebox{-1pt}{4}} \langle v \rangle_\R = \langle w \rangle_\R$\\
\textcircled{\raisebox{-1pt}{1}}\\
$v,w$ linear abhängig $\rightarrow \exists a_1,a_2\in \R,$ nicht beide = 0, $a_1v + a_2w = \sigma$. Dann beide $(a_1,a_2) \ne 0\\
a_1v = -a_2w \mid \cd \frac1{a_1}\\
v = - \frac{-a_2}{-a_1}w \in \langle w \rangle_\R  \textcircled{\raisebox{-1pt}{2}}\\
\textcircled{\raisebox{-1pt}{2}}\\
v \in \langle w \rangle_\R$ dass hei\ss t $v = aw$ für ein $a \in \R$ Dann $a \ne 0$, da $v \ne \sigma$. $w = \frac1a \cd v \in \langle v \rangle_\R \textcircled{\raisebox{-1pt}{3}}$\\
\textcircled{\raisebox{-1pt}{3}}\\
$w = bv$ für ein $b \in \R b \ne 0$,da $w \ne \sigma.\\
aw \in \langle w \rangle_\R \Rightarrow aW = (ab)v \in \langle v\rangle_\R\\
\langle w \rangle_\R \subseteq \langle v \rangle_\R$\\
$w = \frac1b w$ Dann analog $\langle v \rangle\R \subseteq \langle w \rangle_\R$\\
Also $ \langle v \rangle\R = \langle w \rangle_\R$\textcircled{\raisebox{-1pt}{4}}\\
\textcircled{\raisebox{-1pt}{4}}\\
$v \in \langle v \rangle_\R = \langle w \rangle_\R$, dass hei\ss t.\\
$ v = a \cd w$ für ein $a \in \R\\
a \cd v + (-a) w = \sigma \Rightarrow v,w$ sind linear abhängig \textcircled{\raisebox{-1pt}{1}}
\item $e_i = \begin{pmatrix}
0\\\vdots\\0\\1\\0\\0\\0
\end{pmatrix} \in \R^n\\
e_1,\ldots,e_n$ sind linear unabhängig.\\
$\sigma = a_1e_1 + \ldots a_ne_n = \begin{pmatrix}
0\\\vdots\\0\\1\\0\\0\\0
\end{pmatrix} + \begin{pmatrix}
0\\\vdots\\0\\a_2\\0\\0\\0
\end{pmatrix} \Rightarrow a_1 = a_2 = \ldots = a_n = 0$
\item $\vektor12{},\vektor{-3}1{},\vektor62{}$ sind linear abhängig $\R^2$: \\
Gesucht sind alle $a_i,b_i \in \R$ mit $a \cd \vektor12{} + b \cd \vektor{-3}1{} + c \cd \vektor62{} = \vektor00{}$\\
Führt auf LGS für a,b,c:\\
$\begin{pmatrix}
1&-3&6&&0\\
2&1&2&&0
\end{pmatrix}\quad\rightarrow\quad \begin{pmatrix}
1&-3&6&&0\\
0&7&-10&&0
\end{pmatrix}\\
c$ ist frei wählbar
\item $\vektor123, \vektor321, \vektor234$ sind linear abhängig in $\R^3$,\\
10.8b) : $\frac54 \vektor123 + \frac14 \vektor321 + (-1)\vektor234 = \vektor000$
\end{enumerate}
\subsection{Satz}\label{sec:0.11}
Seien $v_1,\ldots,v_n \in \R^n$
\begin{enumerate}[a)]
\item $v_1,\ldots,v_m$ sind linear abhängig \textcircled{\raisebox{-1pt}{1}}\\
$\Leftrightarrow \exists i \ldots v_i = \sum\limits^{m}_{\substack{j=1\\j\ne i} }b_jv_j \textcircled{\raisebox{-1pt}{2}} \\\Leftrightarrow \exists i : \langle v_1,\ldots,v_m \rangle_\R = \langle v_1, \ldots v_{i-1}, v_{i+!},\ldots,v_m\rangle_\R$\textcircled{\raisebox{-1pt}{3}}
\item $v_1,\ldots,v_m$ sind linear unabhängig $\Leftrightarrow$ Jedes $v \in \langle v_1,\ldots,v_m\rangle_\R$ lässt sich auf \emph{genau eine} Weise als Linearkombination von $v_1,\ldots,v_m$ schreiben.
\item Sind $v_1,\ldots,v_m$ linear unabhängig und es existiert $v \in \R^n$ mit$ v \ne \langle v_1,\ldots,v_m\rangle_\R$ dann sind auch $v_1,\ldots,v_m,v $ linear unabhängig
\end{enumerate}
\begin{proof}
a) \textcircled{\raisebox{-1pt}{1}} $\Rightarrow$ \textcircled{\raisebox{-1pt}{2}}\\
$v_1,\ldots v_m$ sind linear abhängig\\
$\Rightarrow \exists a_1,\ldots,a_m$ nicht alle = 0,\\
$a_av_i + \ldots + a_mv_m = 0$\\
Sei $a_i \ne 0\\
a_iv_i = \sum\limits_{\substack{j =1\\j\ne i}}^{m} -a_jv_j$\\
$\phantom{a_i}v_i = \sum\limits_{\substack{j =1\\j\ne i}}^{m} -\frac{a_j}{a_i}v_j$ \textcircled{\raisebox{-1pt}{2}}\\
\textcircled{\raisebox{-1pt}{2}} $\Rightarrow \textcircled{\raisebox{-1pt}{3}}$\\
Klar : $\langle v_1,\ldots v_{i-1},v_{i+1},v_m \rangle_\R \subseteq \langle v_1, \ldots ,v_m \rangle_\R$\\
Zeige $\supseteq$\quad $v  = \langle v_1,\ldots, v_m \rangle_\R$, d.h\\
$v = \sum\limits_{j=1}^{m} a_jv_j = \sum\limits_{\substack{j=1\\j\ne i}}^{m} a_jv_j + a_i (\sum\limits_{\substack{j=1\\j\ne i}}^{m} b_jv_j) = \sum\limits_{\substack{j=1\\j\ne i}}^{m} (a_j + a_i b_j)v_j \in \langle v_1,\ldots v_{i-1},v_{i+1}.\ldots,v_m \rangle_\R \textcircled{\raisebox{-1pt}{2}}$
\textcircled{\raisebox{-1pt}{3}} $ \Rightarrow \textcircled{\raisebox{-1pt}{1}}\\
v_i \in \langle v_1 \ldots v_m \rangle_\R = \langle v_1 \ldots v_{i-1}, v_{i+1}, \rangle_\R$, dass hei\ss t es existiert \\
$a_1, \ldots a_{i-1},a_{i+1},\ldots a_{m} \in \R$ mit \[ v_i = \sum_{\substack{j=1\\j \ne i}}^{m} a_jv_j \]
$\Rightarrow \sigma = a_1 + v_1 + \ldots + a_{i-1}v_{i-1} + (-1)v_i + a_{i+1}v_{i+1} + \ldots a_mv_m \qquad v_1\ldots v_m $ linear abhängig 
%TODO Beweis b und c
\end{proof}
\subsection{Satz}\label{sec:0.12}
Sind $v_i, \ldots, v_{n+1} \in \R^n$, so\\
sin $v_i,\ldots, v_{n+1}$ linear abhängig.\\
(Insbesondere ist $m > n$ und $v_i,v_m \in \R^n$, so sind $v_1,\ldots,v_m $ linear abhängig)
\begin {proof}
Suche alle $a_1,\ldots,a{n+1} \in \R$ mit $a_iv_1 + \ldots a_{n+1}v_{n+1} = \vektor0\dots0$\\
Führt zu LGS für $a_1,\ldots, a_{n+1}$ mit Koeffizientenmatrix $(v_1,\ldots, v_2, \ldots, v_{n+1}) =A$\\
Frage : Hat $ A \cd \vektor{a_i}\vdots{a_{n+1}} = \vektor0\vdots0 \in \R^n$ nicht triviale Lösung?\\
Gau\ss\ :\\
$\left(\raisebox{-0.5cm}{\scalebox{4}{A}} \vektor0\vdots0 \longrightarrow \right)$
%TODO Rest von 10.12
\end {proof}
\subsection{Definition}\label{sec:0.13}
Sei $\U$ ein Unterraum von $\R^n$\\
$B \subseteq \U$ hei\ss t Basis von $\U$ falls:
\begin{enumerate}[(1)]
\item $\langle B \rangle_\R = U$
\item B ist linear unabhängig 
\end{enumerate}
($\U = \{\sigma\}, B = \varnothing$)
\subsection{Beispiel}\label{sec:0.14}
\begin{enumerate}[a)]
\item $e_1,\ldots,e_n$ ist Basis von $\R^n$ (kanonische Basis)\\
$e_1 = \begin{pmatrix}
0\\\vdots\\1\\0\\0
\end{pmatrix}\leftarrow i\\
\vektor{a_i}\vdots{a_n} = \sum_{i=1}^{n} a_ie_i$
\item $\vektor12{},\vektor32{}$ ist Basis von $R^2$:\\
Sei $\vektor{x}{y}{}\in R^2$. Gesucht: $a,b \in \R$ mit $a\vektor12{}+b\vektor32{} =\vektor{x}{y}{}$\\
LGS mit variabler rechter Seite\[
\begin{matrix}
1a &+3b&=x\\
2a &+2b&=y
\end{matrix}\]
Gau\ss:\medskip\\
$\begin{pmatrix}
1&3&&x\\
2&2&&y
\end{pmatrix}\quad\rightarrow\quad\begin{pmatrix}
1&3&&x\\
0&-4&&y-2x
\end{pmatrix}$\\
Eindeutige Lösung: $b = -\frac14y+\frac12x\quad a = x-3b = x +\frac34y - \frac32x = -\frac12x + \frac34y$\\
z.B $\vektor{x}{y}{} = \vektor10{}\\
\vektor10{} = -\frac12\vektor12{}+ \frac12\vektor32{}
\R^2 \langle \vektor12{},\vektor32{}\rangle\\
\vektor12{},\vektor32{}$ sind linear unabhängig nach \ref{sec:0.10}c)\\
$\left\{\vektor12{},\vektor32{}\right\}$ Basis.
\item $\U = \left\langle \vektor123,\vektor321,\vektor234\right\rangle_\R\\
\vektor234 = \frac54 \vektor123 + \frac14\vektor321\\
\U = \left\langle \vektor123,\vektor321\right\rangle_\R\\
\vektor123,\vektor321$ linear unabhängig (\ref{sec:0.10}c))\\
$\left\{ \vektor123, \vektor321 \right\}$ Basis von $\U$
\end{enumerate}
\subsection{Satz}\label{sec:0.15}
Jeder Unterraum $\U$ des $\R^n$ besitzt eine Basis.
\begin{proof}
Ist $\U = \{\sigma\}$,so $b = \varnothing$.\\
Sei also $\U \ne \{\sigma\}$.\\
$v_1$ ist linear unabhängig.\\
$\langle v_1\rangle_\R \subseteq \U.$\\
Ist $\U = \langle v_1 \rangle_\R$, so ist $\{ v_1 \}$ Basis von $\U$\\
Ist $\langle v_1 \rangle_\R \subsetneq \U$.\\
Sei $v_2 \in \U \setminus \langle v_1 \rangle_\R$.\\
Nach \ref{sec:0.11}c) ist $\{ v_1,v_2 \}$ linear unabhängig. Ist $\rangle v_1,v_2 \rangle = \U$, so ist $\{ v_1.v_2 \}$ Basis von $\U$.\\
Ist $\langle v_1, v_2 \rangle_\R \subsetneq U$ so wähle $v_3$ usw.\\
Es existiert $m \ne n$ mit $\langle v_1,\ldots v_m\rangle_\R = \U$ und $v_1\ldots,v_m$ sind linear unabhängig.\\
(Denn noch \ref{sec:0.12} gibt es im $\R^n$ keine n+1 linear unabhängige Vektoren)
\end{proof}
\subsection{Satz}\label{sec:0.16}
Je zwei Basen $B_1,B_2$ eines Unterraums $\U$ des $\R^n$ enthalten die gleiche Anzahl von Vektoren $\abs{B_1} = \abs{B_2}$.\\
Insbesondere:\\
Je zwei Basen des $\R^n$ enthalten n Vektoren
\subsection{Definition}\label{sec:0.17}
Ist $\U$ Unterraum von $\R^n$, B Basis von $\U,\, \abs{B} = m$.\\
Dann ist $m$ die \emph{Dimension} von $\U$, $\dim(u) = m.$\\
$\dim(\R^n) = n,\, \dim(\U) \ne n$.
\subsection{Satz (Basisergänzungssatz)}\label{sec:0.18}
Sei $\U$ Unterraum der $\R^n, M \subseteq \U$ eine Menge m linear unabhängiger Vektoren. Dann lässt sich M zu einer Basis von $\U$ ergänzen.
\begin{proof}
Analog zu \ref{sec:0.15}
\end{proof}
\subsection{Korollar}
Ist $\U$ Unterraum des $\R^n$ und $\dim(\U) = n$, dann ist $\U = \R^n$
\begin{proof}
Sei B Basis von $\U$, also $\abs{B} = n$.\\
Nach \ref{sec:0.18} (dort mit $\U = \R^n, M = B$) lässt sich B zu Basis B' von $\R^n$ ergänzen.\\
$\dim(\R^n) = n \Rightarrow \abs{B'} = n.$\\
Also B = B'\\
$\R^n = \langle B'\rangle_\R = \langle B \rangle_\R = \U$
\end{proof}
\subsection{Definition}
Ist $\U$ Unterraum von $\R^n$, B = ($u_1\ldots,u_m$) eine geordnete Basis von $\U$. Nach \ref{sec:0.11}b), lässt sich jeder Vektorraum $\U = \langle B\rangle_\R$ \emph{eindeutig} als Linearkombination \[ \U = \sum_{i=1}^{m} a_iu_i\quad,a_i \in \R \] schreiben.\\
$(a_1\ldots,a_m)$ hei\ss en \emph{Koordinaten} von u bzgl. der Basis B.
\subsection{Beispiele}
\begin{enumerate}[a)]
\item B($e_1\ldots,e_m$) kanonische Basis von $\R^n$.\\
Koordinaten von $\vektor{a_1}\vdots{a_n} \in \R^n$ bzgl. B:\\
$(a_1\ldots,a_n)$ \emph{kartesische} Koordinaten. \hfill (\emph{Rene Descartes, 1596-1650}) 
\end{enumerate}
\newpage
\begin{center}
\Huge Anfang des WS 2015/16
\end{center}\section{Algebraische Strukturen}
\marginpar{13.10.2015}
\subsection{Definition}\label{sec:1.1}
Sei $X \neq \varnothing$. Eine \index{Verknüpfung}\emph{Verknüpfung} auf $X$ ist \index{Abbildung}: \\ $\begin{cases}
X \times X &\longrightarrow X\\
(a,b) &\longrightarrow a \star b
\end{cases}$ ('Produkt' von a und b)\\
$\star$ ist Platzhalter für andere \index{Verknüpfungssymbole}Verknüpfungssymbole, die in speziellen Beispielen auftreten können.
\subsection{Beispiele}\label{sec:1.2}
\begin{enumerate}[a)]
\item Addition $+$ und Multiplikation $\cd$ sind Verknüpfungen auf $\N, \Z, \mathbb{Q}, \R, \mathbb{C}$.
Multiplikation ist \emph{keine} Verknüpfung auf der Menge der negativen ganzen Zahlen.
\item Division ist keine Verknüpfung auf $\N$. Division ist Verknüpfung auf $\mathbb{Q} \setminus \{0\}, \R \setminus \{0\}, \C \setminus \{0\}$
\item $\Z_n : = \{ 0,1,\ldots,n-1 \}$ \hfill $(n \in \N)\\
a \oplus b : = (a + b) \mod n \in \Z_n\\
a \circledcirc b := (a \cd b) \mod n \in \Z_n$\\
Verknüpfungen auf $\Z_n\\
n = 7:\qquad 5 \circledcirc 6 = 2\\
\phantom{n = 7::}\qquad 5 \oplus 6 = 4\\
n = 2:\qquad \Z_n = \{0,1\}\\
\phantom{n = 2::}\qquad0 \oplus 0 = 0, 1 \oplus 0 = 1,0 \oplus 1 = 1, 1 \oplus 1 = 0\\
\phantom{n = 2::}\qquad\circledcirc = \cdot$
\item M Menge, X = Menge aller Abbildungen $M \longrightarrow M$. Verknüpfung auf X: Hintereinanderausführung von Abbildungen: $\circ$\\
$(f, g): M \longrightarrow M$, So $f\circ g:\:M \to M\\
(f \circ g)(m) = f(g(m)) \in M, m \in M$\\
Im Allgemeinen ist $g \circ f \ne f \circ g$
\item $X = \{0,1\}$\\
2-stellige Aussagen, Junktoren wie $\land,\lor,$\textsf{ XOR }$,\Rightarrow,\Leftrightarrow$ hei\ss en Verknüpfungen auf $X$.
0 entspricht f, 1 entspricht w.\\
$0 \lor 0 = 0, 1 \lor 0 = 1, 0 \lor 1 = 1, 1 \lor 1 = 1\\
0 \land 0 = 0, 0 \land 1 = 0, 1 \land 0 = 0, 1 \land 1 =1$ (= 'Multiplikation')\\
$0 \textsf{ XOR } 0 = 0, 1 \textsf{ XOR } 0 = 1, 0 \textsf{ XOR } 1 = 1, 1 \textsf{ XOR } 1 = 0$ (= Addition mod 2)
\item $X = M_n(\R)$ = Menge der  $n \times n$- Matrizen über $\R$.\\ \index{Matrizenaddition}Matrizenaddition ist Verknüpfung auf $X$.\\
\index{Matrizenmultiplikation}Matrizenmultiplikation ist Verknüpfung auf $X$.
\item $M$ Menge. $X$ Menge aller endlichen Folgen von Elementen aus M ('Wörter' über M).\\
Verknüpfung: Hintereinanderausführung zweier Folgen (\index{Konkatenation}Konkatenation).\\
$M = \{0,1\}, w_1 = 1101, w_2 = 001\\
w_1w_2 = 110111\\
w_2w_1 = 0011101$\\
\end{enumerate}
\subsection{Definition}\label{sec:1.3}
Sei $X \ne 0$ eine Menge mit Verknüpfung $\star$.
\begin{enumerate}[a)]
\item $X$, genauer $(X,\star)$ ist \index{Halbgruppe}\emph{Halbgruppe}, falls
($a \star b) \star c = a \star (b \star c)$ für alle $a,b,c \in X.$ (\index{Assoziativgesetz}\emph{Assoziativgesetz})
\item $(X,\star)$ hei\ss t \index{Monoid}\emph{Monoid}, falls $(X,\star)$ Halbgruppe ist und ein $e \in X$ existiert mit $e \star a = a$ und $a \star e = a$ für alle $a \in X$. e hei\ss t \index{neutrales Element}\emph{neutrales Element} (später, e ist eindeutig bestimmt).
\item Sei $(X,\star)$ ein Monoid. Ein Element $a \in X$ hei\ss t \index{invertierbar}\emph{invertierbar}, falls $b \in X$ existiert (abhängig von a) mit $a \star b = b \star a = e$. b hei\ss t \emph{inverses Element}\index{inverses Element} (das \emph{Inverse}\index{Inverse}) zu a (später: wenn b existiert, so ist es eindeutig bestimmt).
\item Monoid $(X,\star)$ hei\ss t \index{Gruppe}\emph{Gruppe}, falls jedes Element in $X$ bezüglich $\star$ invertierbar ist.
\item Halbgruppe, Monoid, Gruppe $(X,\star)$ bezüglich kommutativ (oder \index{abelsch}\emph{abelsch}) falls $a \star b = b \star a $ für alle $a,b \in X$ (\index{Kommutativgesetz}Kommutativgesetz). \\(Nach: Abel, 1802-1829)
\end{enumerate}
\marginpar{14.10.2015}
<<<<<<< HEAD
\subsection{Bermerkung}\label{sec:1.4}
In Halbgruppe liefert jede sinnvolle Klammerung eines Produktes mit endlich vielen Faktoren das gleiche Element.
=======
\subsection{Bemerkung}\label{sec:1.4}
In Halbgruppe liefert jede sinnvolle Klammerung eines Produktes mit endlich vielen Faktoren das gleiche Element.\\
>>>>>>> dbd12bd4a4b0c211913e023b2cc43fbc6e314244
\begin{equation}
\tag{n = 4}
(a \star (b \star c)) \star d \underset{\text{AG\footnotemark[1]}}{=} ((a \star b)\star c)\star d \underset{\text{AG\footnotemark[1]}}{=} (a \star b) \star (c \star d) \underset{\text{AG\footnotemark[1]}}{=} a \star (b (c \star d)) \underset{\text{AG\footnotemark}}{=} a \star ((b \star c)\star d)
\end{equation}\footnotetext[1]{Assoziativgesetz}
Klammern werden daher meist weggelassen.\\
$a^n = \underset{\substack{\longleftarrow n \longrightarrow\\
n \in \R}}{a\star\ldots\star a}$ ''Potenzen eindeutig definiert''
\subsection{Proposition}\label{sec:1.5}
\begin{enumerate}[a)]
\item In einem Monoid $(X,\star)$ ist das neutrale Element eindeutig bestimmt.
\item Ist $(X,\star)$ Monoid und ist $a \in X$ invertierbar, so ist das Inverse zu a eindeutig bestimmt. Bezeichnung: $a^{-1}$
\item Ist $(X, \star)$ Monoid und wenn $a,b \in X$ invertierbar sind, so auch $a \star b.\\
(a \star b)^{-1} = b^{-1} \star a^{-1}$
\item Die Menge der invertierbaren Elemente in einem Monoid $(X, \star)$ bilden bezüglich $\star$ eine Gruppe.
\end{enumerate}
\begin{proof}
a) Angenommen: $e_!,e_2$ sind neutrale Elemente. Dann:
\begin{align}
e_1 = e_1 \star e_2 = e_1 \star e_2 = e_2 \qquad{\text{\Lightning}}\notag
\end{align}
b) Angenommen a hat 2 inverse Elemente $b_1, b_2$ also.\\
\begin{align}
a \star b_1 &= e, b_2 \star a = e \notag\\
b_1 = e \star b_1 = (b_2 \star a) \star b_1 &= b_2 \star (a \star b_1) = b_2 \star e = b_2 \qquad \text{\Lightning} \notag
\end{align}
c) $$(a \star b)\star (b^{-1} \star a^{-1}) = a \star (b \star b^{-1}) \star a^{-1} = a \star e \star a^{-1} = e$$\\
Analog: $(b^{-1} \star a^{-1})\star (a \star b) =e $\\
Also: $(a \star b)^{-1} = b^{-1} \star a^{-1}$\\
d) $\mathcal{I}$= Menge der inversen Elemente in $(X, \star)$,\\ $e \in \mathcal{I}$, dann $e \star e = e,$ dass hei\ss t $e^{-1} = e, \star$ ist Verknüpfung auf $\mathcal{I}$.\\
Zu zeigen: $a,b \in \mathcal{I} \Rightarrow a \star b \in \mathcal{I}$ Folgt aus c).\\
Assozativgesetz gilt in $\mathcal{I}, a \in \mathcal{I} \Rightarrow a^{-1} \in \mathcal{I}$, denn $(a^{-1})^{-1} = a$
\end{proof}
\emph{Bemerkung}: Multiplikation mit $a^{-1}$ macht Multiplikation mit $a$ (Verknüpfung) rückgängig.
\subsection{Beispiel}\label{sec:1.6}
\begin{enumerate}[a)]
\item $\N,\Z,\mathbb{Q},\R,\C$ sind Halbgruppen bezüglich +.\\
$\Z,\mathbb{Q},\R,\C$ sind bezüglich + Monoide mit neutralen Element 0.\\
$\N = \{ 1,2,\ldots \}$ ist kein Monoid bezüglich +, aber $\N_0$.\\
$\Z,\mathbb{Q},\R,\C$ sind Gruppen bezüglich +. Inverses Element zu $a : -a\\
\N$ ist keine Gruppe bezüglich +, Inverse Elemente in $\N_0:\: \{0\}$
\item $\N,\Z,\mathbb{Q},\R,\C$ sind Monoide bezüglich $\cdot$ (neutrales Element 1). Keine Gruppen (in $\Z, \mathbb{Q},\R,\C$ ist 0 nicht invertierbar).\\
$\mathbb{Q}\setminus\{0\},\R \setminus \{0\}, \C \setminus \{0\}$ Gruppen.\\
Invertierbare Elemente in $\Z:\:: \underset{\substack{\uparrow\\\text{Eigenes Inverses}}}{\{1,-1\}}$ $\leftarrow$ Gruppe bezüglich $\cdot$
\item M Menge.\\
$X$ = Menge aller Abbildungen $M \longrightarrow M$ mit Hintereinanderausführung $\circ$ als Verknüpfung.\\
Monoid, neutrales Element. $id_M$\\
$f \circ id_M = f = id_M \circ f\\
id_M(m) = m$ für alle $m \in M$.\\
Invertierbar sind genau die bijektiven Abbildungen $M \longrightarrow M$, Inverse = Umkehrabbildung.\\
$f : M \longrightarrow M$ bijektiv\\
$\phantom{f : }f \circ f^{-1} = f^{-1} \circ f = id_M$\\
\Nameref{sec:1.5} d): Die bijektive n Abbildung, $ M \longrightarrow M$ bilden bezüglich $\circ$ eine Gruppe	
\item $M$ = Menge z.B $\{0,1\}$, x Menge aller endlichen Folgen über $m$.Halbgruppe mit Verknüpfung Konkatenation . Nimmt man die leere Folge mit hinzu, ist es das neutrale Element. Dann: Monoid.
\item $M_n(\R)$ Menge der Matrizen über $\R$.\\
Addition: neutrales Element $0-Matrix$, Inverse zu A ist -A. ($M$,Addition) ist Gruppe\\
Multiplikation: $(A \cd B) \cd C = A \cd (B \cd C)$ Halbgruppe mit neutralem Element $I_m$
\item $n \in \N\qquad \Z_n = \{0,\ldots,n-1 \}$\qquad Verknüpfung $\oplus\\
a \oplus b = a + b \mod n\\
(\Z_n, \oplus )$ ist Gruppe.\\
Assoziativgesetz: $a,b,c \in \Z_n$\\
$\begin{array}{lcl}
(a \oplus b)\oplus c &=& (a+b \mod n)\mod n\\
&\underset{\text{Mathe I}}{=}& ((a + b) + c) \mod n\\
&=& (a + (b + c)) \mod n\\
&\underset{\text{Mathe I}}{=}& (a + (b + c)\mod n)\mod n\\
&=& (a + (b \oplus c)) \mod n\\
&=& (a \oplus (b \oplus c))
\end{array}$\\
0 ist neutrales Element bezüglich $\oplus$\\
0 ist sein eigenes Inverse.\\
$1 \leq i \leq n \qquad n - i \in \Z_n$ Inverses zu i\\
$\phantom{=}i \oplus (n - i)\\
=(i+ (n-i))\mod n
=n \mod n = 0$
\item $n \in \N, \Z_0$\qquad Verknüpfung $\circledcirc$\qquad $\mathit{n > 1}$\\
$a \circledcirc b = a \cdot b \mod n\\
\mathit{(\Z_n \circledcirc)}$ \emph{ist Monoid}\\
Assoziativgesetz wie bei $\oplus$.\\
1 ist neutrales Element bei $\circledcirc$ Keine Gruppe bezüglich $\circledcirc$, denn 0 hat kein Inverses 
\end{enumerate}
\subsection{Satz}\label{sec:1.7}
Sei $n \in \N, n > 1 $
\begin{enumerate}[a)]
\item Die Elemente in $(\Z_n, \circledcirc)$, die invertierbar bezüglich $\circledcirc$ sind, sind genau diejenigen $a \in \Z_n$ mit ggT$(a,n) = 1$.\\
Für solche a bestimmt man das Inverse folgenderma\ss en:\\
Bestimme $s,t \in \Z$ mit $s \cd a + t \cd n =1$ \hfill(\index{Erweiterter Euklidischer Algorithmus}Erweiterter Euklidischer Algorithmus)\\
Dann ist $a^{-1} = s \mod n$
\item ${\Z_n}^* := \{a \in \Z_n :$ ggT$(a,n)=1 \}$ ist Gruppe bezüglich $\circledcirc$.\\
$\abs{{\Z_n}^*}=: \varphi(n)$ \index{Euler'sche $\varphi$-Funktion}\qquad\emph{Euler'sche $\varphi$-Funktion}\hfill(Leonard Euler 1707-1783)
\item Ist p eine Primzahl so ist $(\Z_p \setminus {0}, \circledcirc)$ eine Gruppe. \emph{Beweis} folgt aus b)
\end{enumerate}
\begin{proof}
a) Angenommen $a \in \Z_n$ invertierbar bezüglich $\circledcirc$\\
D.h es existiert $b \in Z_n$ mit $a \circledcirc b =1$\\
$a \cdot b \mod n = 1$, d.h es existiert $k \in \Z$ mit $a \cd b = 1 + k \cd n, 1 = a \cd b - k \cd n$\\
Sei $d =$ggT$(a.n)$:\\$
\begin{array}{lll}
&d \mid a &\Rightarrow d \mid a \cd b\\
&d \mid n &\Rightarrow d \mid k \cd n\\
\Rightarrow& d \mid a \cd b &- k \cd n = 1\\
\Rightarrow& d =1 & \mathit{ggT}(a,n) = 1.
\end{array}$\\
Umgekehrt sei $a \in \Z_n$ mit ggT($a,n$) = 1\\
EEA liefert $s,t \in \Z$ mit $s \cd a + t \cd n = 1.\\
\begin{array}{cll}
& (s \mod n) \circledcirc a &= ((s \mod n)\cd a) \mod n\\
\underset{\text{Mathe I}}{=}&(s \cd a)\mod n &= (1-t \cd n) \mod n\\
=& (1 - \underbrace{(t \cd n)\mod n)}_{=0} \mod n &= 1 \mod n = 1
\end{array}\\
$b) \Nameref{sec:1.5} d)
\end{proof}
\subsection{Beispiel}\label{sec:1.8}
$n = 24,\quad a =7$ ist invertierbar in $(Z_{24}, \circledcirc)$\\
EEA:\\
\phantom{EEA:}$ 1 = (-2) \cd 24 + 7 \cd 7\\
a^{-1} = 7 \mod 24 = 7 = a$
\subsection{Beispiel}\label{sec:1.9}
Sei $M = \{1,\ldots,n\}$\\
Die Menge der bijektiven Abbildungen auf $M$ (\emph{Permutationen}\index{Permutationen}) bilden nach \ref{sec:1.6}c) eine Gruppe bezüglich Hintereinanderausführung $\circ$.\\
Bezeichnung: $S_n$\index{systematische Gruppe} \emph{systematische Gruppe von Grad n}\\
Es ist $\abs{S_n} = n!$\hfill(Mathe I)\\
z.B : $\pi = \begin{pmatrix}
1&2&3\\
3&2&1
\end{pmatrix} \in S_3\\$
\phantom{z.B : }$\pi^{-1} = \begin{pmatrix}
1&2&3\\
3&2&1
\end{pmatrix} = \pi\\
\phantom{z.B : }\varrho = \begin{pmatrix}
1&2&3\\
3&1&2
\end{pmatrix} \in S_3\\
\phantom{z.B : }\varrho^{-1} = \begin{pmatrix}
1&2&3\\
2&3&1
\end{pmatrix}\\
\phantom{z.B : }\varrho \circ \varrho^{-1} = \mathit{id}\\
\phantom{z.B : }\pi \circ \varrho = \begin{pmatrix}
1&2&3\\
1&3&2
\end{pmatrix}\\
\phantom{z.B : }\varrho \circ \pi  = \begin{pmatrix}
1&2&3\\
2&1&3
\end{pmatrix}\\
S_n$ ist für $n \geq$ 3 nicht abelsch (nicht kommutativ)
\subsection{Satz (Gleichungslösen in Gruppen)}\label{sec:1.10}
Sei $(G, \cd)$ eine Gruppe $a,b \in G$ (in allgemeinen Gruppen schreibt man Verknüpfungen oft als $\cd$ statt $\star$, oft auch ab statt $a\cd b$)
\begin{enumerate}[a)]
\item Es gibt genau ein $x \in G$ mit $ax =b$ (nämlich $x = a^{-1}b$)
[ ''Teilen durch'' $a$ von links = Multiplikation von links mit $a^{-1}$ ]
\item Es gibt genau ein $y \in G$ mit $ya = b$ (nämlich $y = ba^{-1})$
\item Ist $ax = bx$ für ein $x \in G$, so ist $a =b$\\
Ist $ya = yb$ für ein $y \in G$, so ist $a =b$
\end{enumerate}
\begin{proof}
a) Setze $x = a^{-1}b \in G$.\\
$a \cd (a^{-1}\cd b) = (a \cd a^{-1}) b = a \cd b = b$
Eindeutigkeit : Sei $x \in G$ mit $ax = b$\\
Multiplikation beide Seiten mit $a^{-1}$,\\
$\mathit{x} = (a^{-1}a)x = \mathit{a^{-1}b}\\
$\\
b) analog\\
c) $ax =bx$ Multiplikation mit $x^{-1}$ Dann a= b
\end{proof}
\subsection{Beispiel}\label{sec:1.11}
\begin{enumerate}[a)]
\item Suche Permutation $\xi \in S_3$ mit $\varrho \circ \xi = \pi $ (vgl. \ref{sec:1.9}).
\Nameref{sec:1.10}a):\\
$\xi = \varrho^{-1} \circ \pi = \begin{pmatrix}
1&2&3\\
2&3&1
\end{pmatrix}\circ \begin{pmatrix}
1&2&3\\
3&2&1
\end{pmatrix}\\
\phantom{\xi = \varrho^{-1} \circ \pi} = \begin{pmatrix}
1&2&3\\
1&3&2
\end{pmatrix}$
\item \ref{sec:1.10}c) gilt in Monoiden, die keine Gruppen sind, im Allgemeinen nicht:\\
Beispiel: ($\Z_0,\circledcirc)\\
2 \circledcirc 3 - 0 = 3 \circledcirc 3$, aber $2\ne 4$
\end{enumerate}
\subsection{Definition}\label{sec:1.12}
\begin{enumerate}[a)]
\item$R \ne \varnothing$ Menge mit 2 Verknüpfungen + und $\cd$ hei\ss t \index{Ring} \emph{Ring}, falls
\begin{enumerate}[(1)]
\item$ (R, +)$ ist kommutative Gruppe (neutrales Element: 0, \emph{Nullelement}\index{Nullelement}, Inverses zu a : $-a\qquad b + (-a) =: b - a)$
\item $(R,\cd)$ ist Halbgruppe
\item $(a+b) \cd c = a \cd c + b \cd c$ und $a \cd (b + c) = a \cd b + a \cd c \hfill (\cd $ vor $+$)\\
\emph{Distributivgesetz}\index{Distributivgesetz} 
\end{enumerate}
\item Ring R hei\ss t \emph{kommutativer Ring}\index{kommutativer Ring} falls $(R,\cd)$ kommutative Halbgruppe ist.
\item Ring R hei\ss t {\em Ring mit Eins}\index{Ring mit Eins}, falls $(R, \cd)$ Monoid, neutrales Element $1 \ne 0$ ({\em Einselement, Eins}\index{Einselement})
\end{enumerate}
\subsection{Beispiele}\label{sec:1.13}
\begin{enumerate}[a)]
\item $(\Z,+,\cd)$ ist kommutativer Ring mit 1, invertierbare Elemente bezüglich $\cd$ sind 1 und $-1$.
\item $(\mathbb{Q},+,\cd),(\R,+,\cd),(\C,+,\cd)$ sind kommutative Ringe mit Eins.\\
Alle Elemente $\ne 0$ sind invertierbar bezüglich $\cd$
\item $n \in \N, n > 1.\\
\Z_n = \{0,\ldots,n-1 \}\\
(\Z_N, \oplus,\circledcirc)$ ist kommutativer Ring mit Eins:\\
Wegen \Nameref{sec:1.6} f),g) sind nur die Distributivgesetz zu zeigen:\\
$\begin{array}{cl}
&(a \oplus b)\circledcirc c = ((a \oplus b) \cd c)\mod n\\
=&(((a+b)\mod n)\cd c)\mod n\\
\underset{\text{Mathe I}}{=}& ((a+b)\cd )\mod n \\
=&(a\cd c + b \cd c)\mod n\\
\underset{\text{Mathe I}}{=}& ((a \cd c)\mod n + (b \cd c)\mod n) \mod n\\
=& a \circledcirc c \oplus b \circledcirc c 
\end{array}$
\item $M_n(\R), n \times n$-Matrizen über $\R$, mit \index{Matrizenaddition}Matrizenaddition + und, \index{Matrizenmultiplikation}Multiplikation $\cd$ ist Ring mit Eins.\\
(Folgt aus Rechenregeln für Matrizen, \href{http://www.ffgti.org/skripte/MatheII.pdf}{Mathe II}) Eins : $E_n\quad n \times n$-Einheitsmatrix\\
Für $n \geq 2$ ist $M_n(\R)$ kein kommutativer Ring
\end{enumerate}
\subsection{Proposition}\label{sec:1.14}
Sei $(R,+,\cd)$ ein Ring. Dann gilt für alle $a,b \in R$.
\begin{enumerate}[a)]
\item $0 \cd a = a \cd 0 = 0$
\item $(-a) \cd b = a \cd (-b) = - (a \cd b)$
\item $(-a) \cd (-b) = a \cd b$
\end{enumerate}
\begin{proof}\ 
\begin{enumerate}[a)]
\item $0 \cd a = (0 + 0)\cd a\underset{\text{DG\footnotemark}}{=} 0 \cd a + 0 \cd a$\\
\footnotetext[2]{Distributivgesetz}Addiere auf beiden Seiten $-(0 \cd a)\\
0 = 0 \cd a + 0 = 0 \cd a$
\item $(-a) \cd b + ab = ((-a)+ a) \cd b = 0 \cd b \underset{\text{a)}}{=} 0\\
\Rightarrow (-a) \cd b = -(ab)$ Analog $a \cd (-b) = -(ab)$
\item $(-a) \cd (-b) \underset{\text{b)}}{=} - (a \cd (-b)) \underset{\text{b)}}{=} -(- (a \cd b)) = a \cd b$
\end{enumerate}
\end{proof}
\subsection{Bemerkung}\label{sec:1.15}
\begin{enumerate}[a)]
\item In einem Ring mit Eins sind 1 und $-1$ bezüglich $\cd$ invertierbar.\\
$1 \cd 1 = 1\quad(1^{-1} =1)\\
(-1)\cd (-1) = 1$\quad (\ref{sec:1.14}c)), dass hei\ss t. $(-1)^{-1} = -1$\\
0 Ist nie bezüglich Multiplikation invertierbar, denn $0 \cd a = 0 \ne 1$.\quad\ref{sec:1.14}a)
\item Es kann sein dass $ 1 = -1$ gilt. Zum Beispiel:\\
$(\Z_2, \oplus, \circledcirc)\quad 1\oplus 1 = 0\quad 1 = -1$
\end{enumerate}
\subsection{Definition}
Ein kommutativer Ring $(R,+,\cd)$ mit Eins hei\ss t \emph{Körper}\index{Körper}, wenn jedes Element $\ne 0$ bezüglich Multiplikation invertierbar ist.
\subsection{Beispiel}
\begin{enumerate}[a)]
\item $\mathbb{Q},\R,\C$ sind Körper, $\Z$ nicht.
\item $(\Z_n,\oplus,\circledcirc)$ ist genau dann ein Körper, wenn n eine Primzahl.\\
$\Z_n$ ist kommutativer Ring mit 1.\\
\Nameref{sec:1.13}c: Die invertierbaren Elemente in $\Z_n$ sind alle $a \in \Z_n$ mit ggT$(a,n)=a$
\end{enumerate}
\subsection{Proposition \index{Nullteilerfreiheit}(Nullteilerfreiheit in Körpern)}\label{sec:1.18}
Ist $K$ ein Körper, $a,b \in K$, mit $a \cd b = 0$, so ist $a = 0$ oder $b = 0$
\begin{proof}\ \\
Sei $a \cd b = 0$ Angenommen $a \ne 0$.
Dann existiert $a^{-1} \in K\\
0 \underset{\text{\ref{sec:1.14}a)} }{=} a^{-1} \cd 0 \underset{\text{Vor.} }{=} a^{-1}(a \cd b) = (a^{-1} \cd a)\cd b = b$
\end{proof} 
\noindent\emph{Beispiel}: $R = (\Z_6,\oplus,\circledcirc)$\\
\phantom{Beispiel:} $2\circledcirc3=0\qquad 2 \ne 0, 3\ne 0$
\subsection{Definition}\label{sec:1.19}
Sei $K$ ein Körper,
\begin{enumerate}[a)]
\item Ein (Formales) \index{Polynom}\emph{Polynom} über $K$ ist ein Ausdruck $f = a_0 + a_1x + a_2x^2 +\ldots+ a_n x^n = \sum\limits_{i=0}^{n}a_ix^i$ wobei $n \in \N_0, a_i \in K$.
(Manchmal $f(x)$ statt $f,\,+$-Zeichen hat zunächst nichts mit einer Addition zu tun.
$a_i$ \index{Koeffizienten}\emph{Koeffizienten} von f\\
Ist $a_i = 0$ so kann man in der Schreibweise von f $0 \cd x^i$ auch weglassen.\\
Statt $a_0x^0$ schreibt man $a_o$, statt $a_1x^1$ schreibt man $a_1x$. Sind alle $a_i =0,$ so $f = 0$, \emph{Nullpolynom} \index{Nullpolynom}.\\ Ist $a_i = 1$, so schreibt man $x^i$ statt $1x^i$
\item Zwei Polynome $f$ und $g$ sind \emph{gleich}, wenn \emph{entweder} $f=0$ und $g=0$ \emph{oder} $f \ne 0$ und $g \ne 0$\\
d.h $f = \sum\limits_{i = 0}^{n} a_ix^i, a_n \ne 0\\
g = \sum\limits_{i = 0}^{m} a_ix^i, b_m \ne 0$\\
und $n=m$ und $a_i = b_i$ für $i=0...n$
\item Menge aller Polynome über $K$. $K[x]$\\
Wir wollen $K[x]$ zu einem Ring machen. Wie?\\
\emph{Beispiel}:$f = 3x^2+2x+1,$\\
\phantom{\emph{Beispiel}:}$g = 5x^3 + x^2 + x \in Q[x]$\\
$f + g = 5x^3 + 4x^2 + 3x + 1\\
f \cd g = (3^x + 2x + 1)\cd(5x^3 + x^2+ x)\\
\phantom{f \cd g} = 15x^5 + 10 x^4 + 5 x^3 + 3x^4 + 2x^3 + x^2 + 3x^2 + 2x^2 + x\\
\phantom{f \cd g} = 15x^5 + 13x^4+10x^3+3x^2+x$
\end{enumerate}
\marginpar{27.10.2015}
\subsection{Satz und Definition}\label{sec:1.20}
$K$ Körper. $K[x]$ wird zu einem kommutativen Ring mit Eins durch folgenden Verknüpfungen.\\
$f = \sum\limits_{i=0}^{n} a_ix^i, g = \sum\limits_{i=0}^{m}b_ix_i$ so \\
$f + g \sum\limits_{i=0}^{\max(n,m)} (a_i+b_i)x^i\\
f \cd g = \sum\limits_{i=0}^{n+m} c_ix^i$,wobei $c_i = \sum\limits_{j = 0}^{i} a_jb_{i-j}$ \hfill(Faltungsprodukt)\\
In beiden Fällen sind Koeffizienten $a_i$ mit $i > n$ bzw. $b_i$ mit $i > m$ gleich 0 zu setzen. Das Einselement ist $1\ (= 1 x^0)$\\ Das Nullelement ist das Nullpolynom.\\
$-f =\sum\limits_{i=0}^{n}(-a_i)x^i\\
(K[x],+,\cd)$ hei\ss t \emph{Polynomring}\index{Polynomring} in einer Variable
\emph{Beweis:} Nachrechnen
\subsection{Bemerkung}\label{sec:1.21}
\begin{enumerate}[a)]
\item $f = \sum\limits_{i=0}^{n} a^ix^i \in K[x],\, a \in K \subseteq K[x]\\
a \cd f = \sum\limits_{i=0}^{n} (a \cd a_i)x^i\\
x \cd f = \sum\limits_{i=0}^{n}a_ix^{i+1} = a_nx^{n+1} + \ldots + a_0x$
\item Das $+-$ Zeichen in der Definition der Polynome entspricht genau der Addition der \emph{Monome}\index{Monome} $a_ix^i$.\\
$(a_0 x^0 \underset{\substack{\uparrow\\\text{Add. aus \ref{sec:1.20} }}}{+} a_1x^1) = a_0x^0 \underset{\substack{\uparrow\\\text{+ aus \ref{sec:1.19} }}}{+} a_1x^1$
\end{enumerate}
\subsection{Definition}\label{sec:1.22}
Sei $0 \ne f \in k[x], f = \sum\limits_{i=0}^{n}a_ix^i, a_n \ne 0$.\\
Dann hei\ss t $n$ der \index{Grad}\emph{Grad} in $f$, $\Grad(f) = n\\
\Grad(0): = -\infty\\
\Grad(f): = 0 :$\emph{Konstante Polynome}\index{Konstante Polynome} $=\ne 0$
\subsection{Satz}\label{sec:1.23}
Sei $K$ ein Körper, $f,g \in K[x]$.\\
Dann ist $\Grad(f\cd g) = \Grad(f) + \Grad(g)$\\
(Konvention: $-\infty + n = n + (- \infty) = (-\infty + \infty)$,\\
Sei $f \ne 0$ und $g \ne 0\\
f = \sum\limits_{i=0}^{n}a_ix^i, a_n \ne 0, n =\Grad(f)\\
g = \sum\limits_{i=0}^{m}b_ix^i, b_m \ne 0, m =\Grad(g)$\\
Koeffizienten von $x^{n+m}$ in $f \cd g : a_nb_m \underset{\ref{sec:1.18}}{\ne} 0$
\subsection{Korollar}\label{sec:1.24}
Sei $K$ ein Körper
\begin{enumerate}[a)]
\item Genau die konstanten Polynome $\ne 0$ sind in $K[x]$ bezüglich $\cd$ invertierbar\\
Insbesondere ist $K[x]$ \emph{kein} Körper
\item Sind $f,g\in K[x]$ mit $f\cd g = 0$, so ist $f = 0$ oder $g = 0$ (\index{Nullteilerfreiheit}Nullteilerfreiheit in $K[x]$)
\item Sind $f,g_1,g_2 \in K[x]$ mit $f\cd g_1$ und ist $f \ne 0$, so ist $g_1 = g_2$
\end{enumerate}
\begin{proof}\
\begin{enumerate}[a)]
\item Sei $f \in K[x]$ invertierbar bezüglich $\cd$. Dann ist $f \ne 0$ und es existiert $g \in K[x]$ mit $f \cd g = 1$.\\
Mit \ref{sec:1.23}:\\
$ 0 = \Grad(1) = \Grad(f \cd g)\\
\phantom{ 0 } = \Grad(f) + \Grad(1).$\\
Also: $\Grad(f) = 0 (= \Grad(g))$\\
Dass hei\ss t $f$ ist konstantes Polynom.\\
Ist umgekehrt $f = a \in L, a \ne 0,$ so $f^{-1} = a^{-1} \in K$
\item Folgt aus \ref{sec:1.23}:
\begin{enumerate}[ \ ]
\item[ \ ] $-\infty = \Grad(0) = \Grad(f \cd g)\\
\phantom{-\infty} = \Grad(f) + \Grad(g)$
\item[$\Rightarrow$]$\Grad(f) = -\infty$ oder $\Grad(g) = -\infty$, d.h $f = 0$, oder $g = 0$
\end{enumerate}
\item $fg_1 = fg_2\\
\Rightarrow 0 = fg_1 - fg_2
\phantom{\Rightarrow 0} = f \cd (g_1 - g_2)$\\
Da $f \ne 0$, folgt mit b)\\
$g_1 - g_2 = 0$, d.h $g_1 = g_2$
\end{enumerate}
\end{proof}
\subsection{Bemerkung}\label{sec:1.25}
\begin{enumerate}[a)]
\item Jedem Polynom $f = \sum\limits_{i = 0}^{n} a_ix^i \in K[x]$\\
kann man eine Funktion $K \to K$ zuordnen. $a \in K \longmapsto f(a) = \sum\limits_{i=0}^{n} a_ia^i \in K$ \\(Polynomfunktion aus Analysis $K = \R$)\\
Aufgrund der Definition von Addition/Multiplikation von Polynomen gilt:\\
$(f+g)(a) = f(a)+ g(a)\\
(f \cd g)(a) = f(a) \cd g(a)$\\
Es kann passieren, dass zwei verschiedene Polynome die gleiche Funktion beschreiben.\\
Z.B $K = \Z_2 = \{0,1\}\\
f = x^2, g =x\\
f \ne g\\
f(1) = 1 = g(1)\\
f(0) = - g(0)$\\
Über unendlichen Körpern passiert das nicht (später)
\item Schnelle Berechnung von $f(a):\\
f = a_0 + a_1x + \ldots + a_nx^n\\
f(a) = a_0 + a (a (a_1 + a(a_2 + \ldots + a(a_{n-1} + aa_n)))$
{\begin{center}\large\em Horner-Schema\end{center}}\index{Horner-Schema}
\end{enumerate}
\subsection{Definition}\label{sec:1.26}
$K$ Körper, $f,g \in K[x]$\\
$f$ \emph{teilt} $g\quad (f\mid g)$ falls $q \in K[x]$ existiert mit $g = q \cd f$ (Falls $g \ne 0 \mod f \mid g$, so ist $\Grad(f) \leq \Grad(g)$ nach \Nameref{sec:1.23})
\subsection{Satz}\label{sec:1.27}
$K$ Körper, $0 \ne f \in K[x], g \in K[x]$\\
Dann existiert eindeutig bestimmte Polynome $q,r$
\begin{align}
g = q \cd f + r\\
\Grad(r) < \Grad(f)
\end{align}
(Beweis WHK, Satz 4.69)\hfill\emph{Division mit Rest}\index{Division mit Rest}
\subsection{Beispiel}\label{sec:1.28}
\marginpar{28.10.2015}
\begin{enumerate}[a)]
\item $g = x^4 + 2x^3 - x +2, f = 3x^2 - 1, f,g \in Q[x]\\
\polyset{style=C, div=:,vars=x}
    \polylongdiv{x^4+2x^3-x+2}{3x^2 - 1}$
\item $g = x^4 - x^2 + 1\quad f = x^2 + x\quad f,g \in \Z_3[x]\\
x^4 + 3x^3 + 1 : x^2 + x = x^2 + 2 x\\
-\underline{(x^4 + x^3)}\\
\phantom{ }\qquad\phantom{-(}\,2x^3 + 2x^2 + 1\\
\phantom{ }\qquad-(\underline{2x^3 + 2x^2})\\
\phantom{ }\qquad\qquad 1 \leftarrow r$
\end{enumerate}
\subsection{Korollar}\label{sec:1.29}
$K$ Körper,$a \in K$.\\
$f \in K[x]$ ist genau dann durch $(x-a)$ teilbar, wenn $f(a) =0$ (d.h $a$ ist Nullstelle von $f$)\\
$[f = g \cd (x-a), q \in K[x]]$
\begin{proof}\ \\
Falls $x -a \mid f$, so existiert $q \in K[x]$ mit $f \underset{\ref{sec:1.25}}{=} q (x-a)$.\\ Dann $f(a) = q(a) \cd \underbrace{(a-a)}_{= 0} = 0$.\\
Umgekehrt: Angenommen $f(a)=0$. Division mit Rest von $f$ durch $x-a:\\
f = q \cd (x-a)  r, q,r\in K[x]\\
\Grad(r) < \Grad(x-a) = 1, r \in K$\\
Zeige: $r=0$.\\
$r = f - q \cd (x-a)$\\
Setze $a \in K$ ein.\\
$r \underset{\ref{sec:1.25}}{=} f(a) - q(a) \cd (a-a) = 0 - 0 = 0\\
f = q \cd (x-a)$
\end{proof}
\subsection{Definition}\label{sec:1.30}
$K$ Körper $a \in K$ hei\ss t\emph{m-fache Nullstelle} von $f \in K[x]$, falls $(x-a)^m \mid f$ und $(x-a)^{m+1} \nmid f.\\
$Dass hei\ss t $f = q \cd (x-a)^m$ und $q(a) \ne 0$
\subsection{Beispiel}\label{sec:1.31}
$x^5 + x^4 + 1 \in \Z_3[x]$\\
In $Z_3$ hat $f$ die Nullstelle 1\\
\Nameref{sec:1.29}: $x-1 (= x + 2)$ teilt $f$\\
Dividiere $f$ durch $x-1:\\
f = (x^4 + 2x^3 + 2x + 2) \cd (x - 1)$
\subsection{Satz}\label{sec:1.32}
$K$ Körper, $f \in K[x], \Grad(f) = n \geq 0$ (dass hei\ss t $f \ne 0$).\\
Dann hat $f$ höchstens $n$ Nullstellen in $K$ (einschlie\ss end Vielfachheit). Genauer: Sind $a_1,\ldots,a_k$ die verschiedenen Nullstellen von $f$, so ist \\$f = g \cd (x-a_1)^{m_1} \cd \ldots \cd (x-a_k)^{m_k}, m_i$ Vielfachheiten der Nullstellen $a_i$, $g$ hat keine Nullstelle in $K6$
\begin{proof}
Durch Induktion nach n.\\
$n = 0:\: f = a_0 \ne 0$, ohne Nullstelle.$\checkmark$.\\
Sei $n > 0$. Behauptung sei richtig für alle Polynome von Grad $<n$.\\
Hat $f$ keine Nullstellen, $g=f\checkmark$\\
Hat $f$ Nullstellen $a_1\ldots,a_k, k \geq 1$\\
so $f = q \cd (x-a_1)^{m-1}$ (nach Definition)
$q(a_1) \ne 0.\\
\Grad(q) \underset{\ref{sec:1.23}}{=}n-m_1 \underset{m_1 > 0}{<} n$\\
Wir zeigen:\\
$q$ hat genau die Nullstellen $a_2,\ldots,a_k$ mit Vielfachheiten $m_2,\ldots,m_k$.\\
Klar: Jede Nullstelle von $q$ ist Nullstelle von $f$, Dass hei\ss t $q$ hat höchstens Nullstellen $a_2,\ldots,a_k$.\\
Diese Nullstellen hat $q$ mit Vielfachheit $0 \geq n_i \geq m_i$, denn $(x-a_i)^{m_i} \vert q \Rightarrow (x-a_i)^{n_i}\mid f$\\
Sei $i \in \{2,\ldots,k\}.$ Es ist $f = s \cd (x - a_i)^{m_i}, s \in K[x], s(a_i) \ne 0\\$
\phantom{Sei $i \in \{2,\ldots,k\}$. Es ist }$q = q_1 \cd (x-a_i)^{n_i},q_1 \in K[x], q(a_i) \ne 0,\hfill ((x-a_i)^0 =1 ) $
\phantom{Sei $i \in \{2,\ldots,k\}$. Es ist }$f=q_1(x-a_1)^{n_i} \cd (x-a_1)^{m_1}$
\Nameref{sec:1.24}c):\\
$s(x-a_i)^{m_i-n_i} =q_1 \cd (x-a_1)^{m_1}$\\
Ist $m_i > n_i$, so ist $m_i - n_i > 0$\\
$0 = s(a_i)(a_i -a_i)^{m_i - n_i} = q(a_i)(a_i - a_i) \ne 0 \Lightning$\\
Dass hei\ss t .$n_i = m_i. i,2\ldots,k\\
q = g (x-a_2)^{m_2}\ldots (x-a_k)^{m_k}$, g ohne Nullstelle in $K$\\
$f = g (x-a_1)^{m_2} \cdots (x-a_2)^{m_1}$\hfill (Nach Induktionsvorsaussetzung)
\end{proof}
\subsection{Korollar}\label{sec:1.33}
$K$ Körper, $f,g \in K[x], m = \max(\Grad(f),\Grad(g)$\\
Gibt es $m+1$ Elemente $a_1,\ldots,a_{m+1} \in K$, paarweise verschieden, mit $f(a_i) = g(a_i), i = 1,\ldots,m+1$ so $f = g$.\\
\emph{Insbesondere}: Ist $K$ unendlich ,$f,g \in K[x]$ mit $f(a) = g(a)$ für alle $a \in K$, so ist $f =g$\\
\begin{proof}
$f -g \in K[x], \Grad(f-g) \leq m.\\
f -g $hat $m+1$ Nullstellen $a_1,\ldots a_{m+1}$\\
\ref{sec:1.32} $f - g = 0, f =g$
\end{proof}
\subsection{Bemerkung}
Über $\mathbb{Q},\R,\Z_p (p$ Primzahl) gibt es Polynome beliebig hohen Grades ohne Nullstellen\\
Über $\mathbb{Q},\R$: $(x^2 +1)^m$ hat $\Grad(2m)$, keine Nullstellen in $\mathbb{Q},\R$\\
über $\Z_p$ z.B $(x^p - x +1)^m$ hat $\Grad pm$, ohne Nullstellen (ohne Beweis)
\subsection{Fundamentalsatz der Algebra}
Ist $ f \in \C[x], f \ne  0$ so ist $(f=a_nx^n+\ldots+a_0)\\
f = a_n (x-c_1)^{m_1} \ldots (x-c_k)^{m_k}, a_n.c_i,\ldots,c_k \in \C$ (Nullstellen mit Vielfachen $m_1,m_2$)\\
$m_1 + \ldots + m_k = \Grad(f)\\
\Grad(f) = n$ $f$ hat $n$ Nullstellen (einschlie\ss end Vielfachheit)
\section{Vektorräume}
\marginpar{3.11.2015}
\subsection{Definition}\label{sec:2.1}
Sei $K$ ein Körper. Ein \emph{K-Vektorraum} \index{K-Vektorraum} V besitzt Verknüpfung $+$ bezüglich derer V eine Kommutative Gruppe ist (Neutrales Element $\sigma$, \emph{Nullvektor}\index{Nullvektor}, Inverses zu $v \in V : -v$). Au\ss erdem existiert Abbildung $K \times V \longrightarrow V\\
(a,v)\longmapsto av, a \in K, v \in V$\\
(\glqq Multiplikation\grqq von Elementen aus $V$, (''Vektoren'') mit Körperelementen (''Skalare'')), so dass gilt:\\ $(a \underset{\text{in $K$ }}{+} b)v = av \underset{\text{in $V$ }}{+} bv$ für alle $a,b \in K,\, v \in V\\
a(v\underset{\text{in $V$ }}{+}w) = av \underset{\text{in $V$ }}{+} aw$ für alle $a \in K,\, v,w \in V\\
\underset{\text{in $K$ }}{(ab)}v = a(\underset{\in V}{bv})$ für alle $a,b\in K, v \in V\\
1v = v$ für alle $v \in V$.
\subsection{Beispiel}\label{sec:2.2}
\begin{enumerate}[a)]
\item $K$ Körper, $n \in \N$\\
$K^n = \left\{ \vektor{a_1}{\vdots}{a_n} : a_i \in K \right\}$ ist K-Vektorraum bezüglich $\vektor{a_1}{\vdots}{a_n} + \vektor{b_1}{\vdots}{b_n} = \vektor{a_1 + b_1}{\vdots}{a_n + b_n}\\
a\vektor{a_1}{\vdots}{a_n} = \vektor{aa_1}{\vdots}{aa_n}$ für alle $a \in K, \vektor{a_1}{\vdots}{a_n},\vektor{b_1}{\vdots}{ab_n} \in K^n$. Raum der \emph{Spaltenvektoren}\index{Spaltenvektoren} der \emph{Länge n} über $K$.\\
Entsprechend: Raum der Zeilenvektor, $\vektor{a_1}{\vdots}{a_n} = (a_1,\ldots,a_n)^t$\\
Für $K = \R : \R^n\\
n = 2,3$ Elemente aus $\R^2,\R^3$, identifizierbar mit Ortsvektor der Ebene oder des 3-dimensionalen Raumes.
\begin{figure}[h!]
\centering
\begin{tikzpicture}
\begin{axis}[
axis x line=center,
axis y line=center,
axis equal,
ymin = -1,
xmin = -3,
ymax = 7,
xmax = 5.5,
xtick ={4},
xticklabels={a},
ytick ={1},
yticklabels={b}]
\addplot [black, mark = *, nodes near coords=,every node near coord/.style={anchor=180}] coordinates {( 4, 3)};
   \draw[->](axis cs:0,0)--(axis cs:3.88,2.88);
       \addplot [black, mark = *, nodes near coords=,every node near coord/.style={anchor=0}] coordinates {( -1, 1)};
              \draw[->](axis cs:0,0)--(axis cs:-0.88,0.88);
\addplot [black, mark = *, nodes near coords=,every node near coord/.style={anchor=0}] coordinates {( 3, 4)};
\addplot[mark=none, black] coordinates {(-1,1) (3,4)};
\addplot[mark=none, black] coordinates {(3,4) (4,3)};
\draw[-> ,red](axis cs:0,0)--(axis cs:2.95,3.88);
\end{axis}
\end{tikzpicture}\end{figure}
\item Sei $K$ ein Körper Polynomring $K[x]$ ist ein K-Vektorraum, bezüglich
\begin{itemize}
\item Addition von Polynomen
\item Multiplikation von Körperelementen mit Polynomen
\[ a \left(\sum\limits_{i=0}^{n} a_ix^i\right):= \sum\limits_{i=0}^{n} \left(aa_i\right) x^i \in K[x] \]
(Multiplikation von Polynomen mit Polynom $\Grad \leq 0$)\\
\ref{sec:2.1} folgt aus den Ringeingenschaften von $K[x]$
\end{itemize}
\item $K$ Körper. $V$ = Abbildung ($K$,$K$) = $\{\alpha : K \to K : \alpha$Abbildung$ \}$ Addition auf V\\
$\alpha + \beta \in V (\alpha + \beta)(x) = \alpha(x) + \beta(x)$ für alle $x \in K$\\
Skalare Multiplikation:\\
$a \in \R, \alpha \in V (a\alpha)(x) = a \cd \alpha(x)$ Für alle $x \in K$\\
Nachrechnen : Damit wird $V$ ein $K$-Vektorraum
\end{enumerate}
\subsection{Proposition}
$K$ Körper, $V, K-VR$
\begin{enumerate}[a)]
\item $a \cd \sigma = \sigma$
\item $0 \cd v = \sigma$
\item $(-1) \cd v = -v$\\
a,b,c Für alle $v \in V$
\end{enumerate}
\subsection{Definition}
$K$ Körper, $V\ K-VR.\\
\varnothing + U \subseteq V$ hei\ss t \emph{Unterraum}\index{Unterraum} (\emph{Untervektorraum} \index{Untervektorraum}, oder \emph{Teilraum} \index{Teilraum}) von $V$, falls $U$ bezüglich Addition auf $V$ und der skalaren Multiplikation mit Elementen aus $K$ selbst $K$ Vektorraum ist.
\subsection{Proposition}
$U$ ist Unterraum von $V\\
\Leftrightarrow$ \begin{minipage}[t]{.5\textwidth}
$$(1)\quad u_1 + 1_2 \in U \text{ für alle }u_1,u_2 \in U $$
$$(2)\quad au \in U \text{ für alle } u \in U, a \in L$$ 
\end{minipage}\\
(Nullvektor in $U$ = Nullvektor in $V$)\\
\begin{proof}
$\Rightarrow \checkmark
\Leftarrow$: Da $U \ne \varnothing$, existiert $u \in U.\\
\sigma = 0 \cd u \in U\\
u \in U \Rightarrow -u = (-1)u \in U$\\
Mit (1): $(U,+)$ ist kommutative Gruppe. Restliche Axiome gelten auch für $U,K$.
\end{proof}
\subsection{Beispiel}
\begin{enumerate}[a)]
\item$V-K-VR$, so ist $V$ Unterraum von $V$.\\
und $\{0\}$ ist Unterraum von $V$ (\emph{Nullraum}\index{Nullraum})
\item Betrachte $K[x]$ als $K-VR$. (\ref{sec:2.2}).\\
Sei $n \in \N_0$.\\
$U = \{ f \in K[x] : \Grad(f) \leq n \}$ Unterraum von $K[x]$
\end{enumerate}
\subsection{Proposition}
Seien $U_1,U_2$ Unterräume von $K$-VR V.
\begin{enumerate}[a)]
\item $U_1 \cap U_2$ ist Unterraum
\item $U_1 + U_2$ := $\{ u_1 + u_2 : u_1 \in u_1 \in U_2, u_2 \in U_2 \}$ ist Unterraum von $V$ (\emph{Summe} von Unterräume)
\item $U_1 + U_2$ ist der kleinste Unterraum von $V$, der $U_1 \cup U_2$ enthält.
\item $ U_1 \cap U_2$ ist im Allgemeinen kein Unterraum.\\
\emph{Beweis}: \ref{sec:0.4}
\end{enumerate} 
\subsection{Definition}
$V\ K$-VR
\begin{enumerate}[a)]
\item $v_1,\ldots,v_m \in V,\,a-i,\ldots a_m \in K$\\
Dann hei\ss t\\
$a_1v_1+\ldots a_mv_m = \sum\limits^{m}_{i=1} a_i v_i \in V$\\
\emph{Linearkombination}\index{Linearkombination} von $v_1,\ldots,v_m$ (mit Koeffizienten $a_1,\ldots,a_m$).\\
$\left\lbrack\right. $ Beachte: Zwei formell verschiedene Linearkombinationen derselben Vektoren können den gleichen Vektor darstellen z.B. in $\R^2:\:\\
1\cd \vektort10 + 2 \cd \vektort01 + 3 \cd \vektort11\\
\left.2 \cd \vektort10 + 3 cd \vektort01 + 2 \cd \vektort11 = \vektort45\right\rbrack$
\item Ist $M \subseteq V$, so ist der von $M$ \emph{erzeugte} oder \emph{aufgespannte Unterraum}\index{aufgespannte Unterraum} $<M>_k$ (oder kurz $(<M>)$ die Menge aller endlichen Linearkombination, die man mit Vektoren aus $M$ bilden kann:\\
$<M>$ = $\{ \sum\limits_{i=1}^{n}a_iv_1 : n \in \N,a_i \in K,v_i \in M \}\\
<\varnothing>_K:= \{\varnothing\}\\
M = \{ v_1,\ldots v_m \} : <M> =: <v_1,\ldots,v_m>$
\item Ist $<M>_K = V$, so hei\ss t M \emph{Erzeugungssystem}\index{Erzeugungssystem}
\end{enumerate}
\subsection{Satz}
$V\ K-$VR, $M \subseteq V$
\begin{enumerate}[a)]
\item $<M>_K$ ist Unterraum von V
\item $<M>_K$ ist der kleinste Unterraum von $V$, der $M$ enthält.\\
Insbesondere: Sind $u_1,u_2$ Unterräume von $V$, so ist $<U_1 \cup U_2>_K = U_1 + U_2$\\
\emph{Beweis}: \ref{sec:10.7}
\end{enumerate}
\subsection{Definition}
$V\ K$-VR $V$ hei\ss t \emph{endlich erzeugt}\index{endlich erzeugt}, falls es eine \emph{endliche} Teilmenge $M \subseteq V$ gibt mit $V = <M>_K$
\subsection{Beispiel}
\begin{enumerate}[a)]
\item $K^n = \left\{ \vektor{a_1}{\vdots}{a_n} : a_i \in K\right\}\\
K^n$ ist endlich erzeugt.\\
$e_1,\ldots, e_n$ \emph{Einheitsvektor}\index{Einheitsvektor}
$e_i = \vektor{0\\\vdots}{1}{\vdots\\0}\leftarrow i\\
K^n = <e_1,\ldots,e_n>_K$, denn $\vektor{a_1}{\vdots}{a_n} = a_1e_1 + \ldots + a_ne_n$
\item $K[x]$ als $K$-Vr ist nicht endlich erzeugt. Angenommen e existiert $f_1,\ldots,f_n \in K[x]$ mit $K[x] = <f_1,\ldots,f_n>_K$.\\
Sei $t,\,\max \Grad(f_i) \in \N_0 \cup \{-\infty\}$\\
Dann haben alle Polynome in $<f_1,\ldots,f_n>_K$ höchstens Grad $t$. Also $x^{t+1} \in K[x] \setminus <f_1,\ldots,f_n>_K \Lightning\\
M = \{1,x,x^2,x^3,\ldots \} = \{ x^i: i \in \N_0 \}\\
K[x] = <M>_K.\qquad f = \sum\limits_{n=0}^{t} a_i x^i$
\item $n \in \N. \qquad U = \{ f \in K[x]: \Grad(f) = n \}$\\
Unterram von $K[x]$, endlich erzeugt
\end{enumerate}
\subsection{Definition}
Sei $V\ K$-VR, $v_1,\ldots,v_m \in V$ hei\ss en \emph{linear abhängig}\index{linear abhängig}, wenn es $a_1,\ldots,a_n \in K$, \emph{nicht alle = 0}, gibt mit \[ a_1v_1 + \ldots + a_m v_m = \sigma \]
(Beachte: Immer mit $0\cd v_1 + \ldots + 0 \cd v_m = \sigma$,
aber bei lineare Abhängigkeit soll es noch eine andere Möglichkeit geben)
Andernfalls nennt man $v_1,\ldots,v_m$ \emph{linear unabhängig}\index{linear unabhängig}:\\
(D.h aus $a_1v_1+\ldots+a_mv_m = \sigma$ folt $a_1 = \ldots = a_m = 0$)\\
Entsprechend: $\{v_1,\ldots,v_m\}$ linear abhängig, linear unabhängig.\\
$\varnothing$ per Definition linear unabhängig.
Klar: Teilmenge von linear unabhängigen Vektoren wieder linear unabhängig
\subsection{Beispiel}
\begin{enumerate}[a)]
\item $\sigma$ ist linear abhängig: $1 \cd \sigma = \sigma$
\item $v,w \in V, v \ne \sigma \ne w.$\\
Wann sind v und w linear abhängig?\\
$v,w$ linear abhängig $\Rightarrow \exists a,b \in K$,nicht beide = 0 mit $a \cd v + b \cd w = \sigma$\\
Angenommen : $a \ne 0\qquad a \cd v = -b \cd w | a^{-1}$ (K Körper)\\
$v = 1 \cd v = (a^{-1}a)v = a^{-1}(av) = a^{-1}(-bw) = (-a^{-1}b)w \in <w>_K = \{ cw: c \in K \}\\
d \in K\\
dv = (-da^{-1}b)w \in <w>_K\\
<v>_K \subseteq <w>_K$\\
Dann auch $b \ne 0$.\\
Angenommen $b = 0,\quad a \cd v = -0 w = \sigma\\
\phantom{Angenommen\ b =} v = a^{-1} \sigma = \sigma \Lightning$
Vertausche Rollen von $v,w : <w>_K \subseteq <v>_K\\
v,w$ linear abhängig $\Leftrightarrow <v>_K = <w>_K$
\begin{proof}
$\Rightarrow \checkmark\\
\Leftarrow v \in <v>_K = <w>_K\\
\Rightarrow v = c \cd w$ für ein $c \in K$.\\
$\Rightarrow \sigma = -v + c \cd w = (-1) v + c \cd w\\
\Rightarrow v,w$ linear abhängig.
\end{proof}
\item $e_1,\ldots e_n \in K^n$ sind linear unabhängig.\\
$\vektor0\vdots0= a_1e_1 + \ldots + a_n e+n = \vektor{a_1}{\vdots}{a_n}\\
\Rightarrow a_1 = \ldots a_n = 0.$
\item $\vektor123,\vektor321,\vektor234 \in \R^3$ linear abhängig, linear unabhängig? Für welche $a,b,c \in \R$ gilt $a\vektor123+b\vektor321+c\vektor234 = \vektor000$?\\
Führt auf LGS für die unbekannten $a,b,c\\
\begin{matrix}
1a&3b&2c&=0\\
2a&2b&3c&=0\\
3a&1b&4c&=0
\end{matrix}\\
$Gau\ss:\\
$\begin{pmatrix}
1&3&2&&0\\
2&2&3&&0\\
3&1&4&&0
\end{pmatrix} \quad \to \quad 
\begin{pmatrix}
1&3&2&&0\\
0&-4&-1&&0\\
0&-8&-2&&0
\end{pmatrix}
\quad \to \quad 
\begin{pmatrix}
1&3&2&&0\\
0&1&0.25&&0\\
0&0&0&&0
\end{pmatrix}\\
c$ frei wählbar, $b = - \frac14 c\qquad a = -3b-2c = -\frac34c -2c = -\frac54c$\\
z.B $c = 4,b =-1,a=-5\\
(-5)\vektor123 + (-1) \vektor321 + 4 \vektor234 = 0$\\
Vektoren sind linear abhängig.
\end{enumerate}
\subsection{Bemerkung}
Man kann auch für unendliche Mengen $M \subseteq V$ lineare Unabhängigkeit definieren.\\
Jede endliche Teilmenge von $M$ ist linear unabhängig. Zum Beispiel $\{ x^i: i \in \N_0 \}$ linear unabhängig in $K[x]$.
\subsection{Satz !!!}\label{sec:2.15}
$V\ K$-VR, $v_1,\ldots,v_m$ sind linear abhängig
\begin{enumerate}
\item$\Leftrightarrow \exists i : v_i = \sum\limits_{\substack{j=1\\j\ne i} }^{m} b_j v_j$ für geignete $b_j \in K\\
\Leftrightarrow \exists i : <v_1.\ldots, v_m>_K = <v_1,\ldots,v_{i-1},v_{i+1},\ldots,v_m>_K$
\item $v_1,\ldots,v_m$ linear unabhängig\\
$\Leftrightarrow$ jedes $v \in <v_1,\ldots v_m$ lässt sich als $v_1,\ldots,v_m$ schreiben.
\item Sind $v_1,\ldots,v_m$ linear unabhängig und ist $V \nin <v_1,\ldots,v_m>_K$, so sind $v_1,v_m,v$ linear unabhängig.
\begin{proof}
Wie in \ref{sec:0.11}, aber $v_1,\ldots,v_m \in V$
\end{proof}
\end{enumerate}
\subsection{Definition}
Sei $V$ endliche erzeugter $K$-VR.\\
Eine endliche Teilmwnge $B \subseteq V$ heißt \emph{Basis}\index{Basis} von V, falls
\begin{enumerate}[(1)]
\item $V \langle B \rangle_K$
\item $B$ linear unabhängig
\end{enumerate}
($V = \{ \sigma \}: \varnothing$ ist Basis von $V$)
\subsection{Beispiel}
nächsten Dienstag
\subsection{Satz (Existenz von Basen)}
Sei $V$ endliches Erzeugter $K$-VR. Dann enthält jedes endliche Erzeugendensystem von $V$ eine Basis vom $V$.
\begin{proof}
Sei $M \subseteq V$ endlich mit $V = \langle M \rangle_K$.
Ist M linear unabhängig, so ist M Basis $\checkmark$\\
ist M linear abhängig, so existier nach \ref{sec:2.15}a)\\
$v \in M$ mit $V = \langle M \rangle_K = \langle M \setminus \{v\} \rangle_K$\\
Da M endlich, endet dieses Verfahren mit Basis
\end{proof}
\printindex
\end{document}
